% Generated by Sphinx.
\def\sphinxdocclass{report}
\newif\ifsphinxKeepOldNames \sphinxKeepOldNamestrue
\documentclass[letterpaper,10pt,spanish]{sphinxmanual}
\usepackage{iftex}

\ifPDFTeX
  \usepackage[utf8]{inputenc}
\fi
\ifdefined\DeclareUnicodeCharacter
  \DeclareUnicodeCharacter{00A0}{\nobreakspace}
\fi
\usepackage{cmap}
\usepackage[T1]{fontenc}
\usepackage{amsmath,amssymb,amstext}
\usepackage{babel}
\usepackage{times}
\usepackage[Sonny]{fncychap}
\usepackage{longtable}
\usepackage{sphinx}
\usepackage{multirow}
\usepackage{eqparbox}

\addto\captionsspanish{\renewcommand{\contentsname}{Conceptos iniciales}}

\addto\captionsspanish{\renewcommand{\figurename}{Figura }}
\addto\captionsspanish{\renewcommand{\tablename}{Tabla }}
\SetupFloatingEnvironment{literal-block}{name=Lista }

\addto\extrasspanish{\def\pageautorefname{página}}

\setcounter{tocdepth}{1}


\title{IvozProvider 1.0 Documentation}
\date{oct 18, 2016}
\release{Oasis}
\author{Irontec}
\newcommand{\sphinxlogo}{}
\renewcommand{\releasename}{Publicación}
\makeindex

\makeatletter
\def\PYG@reset{\let\PYG@it=\relax \let\PYG@bf=\relax%
    \let\PYG@ul=\relax \let\PYG@tc=\relax%
    \let\PYG@bc=\relax \let\PYG@ff=\relax}
\def\PYG@tok#1{\csname PYG@tok@#1\endcsname}
\def\PYG@toks#1+{\ifx\relax#1\empty\else%
    \PYG@tok{#1}\expandafter\PYG@toks\fi}
\def\PYG@do#1{\PYG@bc{\PYG@tc{\PYG@ul{%
    \PYG@it{\PYG@bf{\PYG@ff{#1}}}}}}}
\def\PYG#1#2{\PYG@reset\PYG@toks#1+\relax+\PYG@do{#2}}

\expandafter\def\csname PYG@tok@gd\endcsname{\def\PYG@tc##1{\textcolor[rgb]{0.63,0.00,0.00}{##1}}}
\expandafter\def\csname PYG@tok@gu\endcsname{\let\PYG@bf=\textbf\def\PYG@tc##1{\textcolor[rgb]{0.50,0.00,0.50}{##1}}}
\expandafter\def\csname PYG@tok@gt\endcsname{\def\PYG@tc##1{\textcolor[rgb]{0.00,0.27,0.87}{##1}}}
\expandafter\def\csname PYG@tok@gs\endcsname{\let\PYG@bf=\textbf}
\expandafter\def\csname PYG@tok@gr\endcsname{\def\PYG@tc##1{\textcolor[rgb]{1.00,0.00,0.00}{##1}}}
\expandafter\def\csname PYG@tok@cm\endcsname{\let\PYG@it=\textit\def\PYG@tc##1{\textcolor[rgb]{0.25,0.50,0.56}{##1}}}
\expandafter\def\csname PYG@tok@vg\endcsname{\def\PYG@tc##1{\textcolor[rgb]{0.73,0.38,0.84}{##1}}}
\expandafter\def\csname PYG@tok@m\endcsname{\def\PYG@tc##1{\textcolor[rgb]{0.13,0.50,0.31}{##1}}}
\expandafter\def\csname PYG@tok@mh\endcsname{\def\PYG@tc##1{\textcolor[rgb]{0.13,0.50,0.31}{##1}}}
\expandafter\def\csname PYG@tok@cs\endcsname{\def\PYG@tc##1{\textcolor[rgb]{0.25,0.50,0.56}{##1}}\def\PYG@bc##1{\setlength{\fboxsep}{0pt}\colorbox[rgb]{1.00,0.94,0.94}{\strut ##1}}}
\expandafter\def\csname PYG@tok@ge\endcsname{\let\PYG@it=\textit}
\expandafter\def\csname PYG@tok@vc\endcsname{\def\PYG@tc##1{\textcolor[rgb]{0.73,0.38,0.84}{##1}}}
\expandafter\def\csname PYG@tok@il\endcsname{\def\PYG@tc##1{\textcolor[rgb]{0.13,0.50,0.31}{##1}}}
\expandafter\def\csname PYG@tok@go\endcsname{\def\PYG@tc##1{\textcolor[rgb]{0.20,0.20,0.20}{##1}}}
\expandafter\def\csname PYG@tok@cp\endcsname{\def\PYG@tc##1{\textcolor[rgb]{0.00,0.44,0.13}{##1}}}
\expandafter\def\csname PYG@tok@gi\endcsname{\def\PYG@tc##1{\textcolor[rgb]{0.00,0.63,0.00}{##1}}}
\expandafter\def\csname PYG@tok@gh\endcsname{\let\PYG@bf=\textbf\def\PYG@tc##1{\textcolor[rgb]{0.00,0.00,0.50}{##1}}}
\expandafter\def\csname PYG@tok@ni\endcsname{\let\PYG@bf=\textbf\def\PYG@tc##1{\textcolor[rgb]{0.84,0.33,0.22}{##1}}}
\expandafter\def\csname PYG@tok@nl\endcsname{\let\PYG@bf=\textbf\def\PYG@tc##1{\textcolor[rgb]{0.00,0.13,0.44}{##1}}}
\expandafter\def\csname PYG@tok@nn\endcsname{\let\PYG@bf=\textbf\def\PYG@tc##1{\textcolor[rgb]{0.05,0.52,0.71}{##1}}}
\expandafter\def\csname PYG@tok@no\endcsname{\def\PYG@tc##1{\textcolor[rgb]{0.38,0.68,0.84}{##1}}}
\expandafter\def\csname PYG@tok@na\endcsname{\def\PYG@tc##1{\textcolor[rgb]{0.25,0.44,0.63}{##1}}}
\expandafter\def\csname PYG@tok@nb\endcsname{\def\PYG@tc##1{\textcolor[rgb]{0.00,0.44,0.13}{##1}}}
\expandafter\def\csname PYG@tok@nc\endcsname{\let\PYG@bf=\textbf\def\PYG@tc##1{\textcolor[rgb]{0.05,0.52,0.71}{##1}}}
\expandafter\def\csname PYG@tok@nd\endcsname{\let\PYG@bf=\textbf\def\PYG@tc##1{\textcolor[rgb]{0.33,0.33,0.33}{##1}}}
\expandafter\def\csname PYG@tok@ne\endcsname{\def\PYG@tc##1{\textcolor[rgb]{0.00,0.44,0.13}{##1}}}
\expandafter\def\csname PYG@tok@nf\endcsname{\def\PYG@tc##1{\textcolor[rgb]{0.02,0.16,0.49}{##1}}}
\expandafter\def\csname PYG@tok@si\endcsname{\let\PYG@it=\textit\def\PYG@tc##1{\textcolor[rgb]{0.44,0.63,0.82}{##1}}}
\expandafter\def\csname PYG@tok@s2\endcsname{\def\PYG@tc##1{\textcolor[rgb]{0.25,0.44,0.63}{##1}}}
\expandafter\def\csname PYG@tok@vi\endcsname{\def\PYG@tc##1{\textcolor[rgb]{0.73,0.38,0.84}{##1}}}
\expandafter\def\csname PYG@tok@nt\endcsname{\let\PYG@bf=\textbf\def\PYG@tc##1{\textcolor[rgb]{0.02,0.16,0.45}{##1}}}
\expandafter\def\csname PYG@tok@nv\endcsname{\def\PYG@tc##1{\textcolor[rgb]{0.73,0.38,0.84}{##1}}}
\expandafter\def\csname PYG@tok@s1\endcsname{\def\PYG@tc##1{\textcolor[rgb]{0.25,0.44,0.63}{##1}}}
\expandafter\def\csname PYG@tok@gp\endcsname{\let\PYG@bf=\textbf\def\PYG@tc##1{\textcolor[rgb]{0.78,0.36,0.04}{##1}}}
\expandafter\def\csname PYG@tok@sh\endcsname{\def\PYG@tc##1{\textcolor[rgb]{0.25,0.44,0.63}{##1}}}
\expandafter\def\csname PYG@tok@ow\endcsname{\let\PYG@bf=\textbf\def\PYG@tc##1{\textcolor[rgb]{0.00,0.44,0.13}{##1}}}
\expandafter\def\csname PYG@tok@sx\endcsname{\def\PYG@tc##1{\textcolor[rgb]{0.78,0.36,0.04}{##1}}}
\expandafter\def\csname PYG@tok@bp\endcsname{\def\PYG@tc##1{\textcolor[rgb]{0.00,0.44,0.13}{##1}}}
\expandafter\def\csname PYG@tok@c1\endcsname{\let\PYG@it=\textit\def\PYG@tc##1{\textcolor[rgb]{0.25,0.50,0.56}{##1}}}
\expandafter\def\csname PYG@tok@kc\endcsname{\let\PYG@bf=\textbf\def\PYG@tc##1{\textcolor[rgb]{0.00,0.44,0.13}{##1}}}
\expandafter\def\csname PYG@tok@c\endcsname{\let\PYG@it=\textit\def\PYG@tc##1{\textcolor[rgb]{0.25,0.50,0.56}{##1}}}
\expandafter\def\csname PYG@tok@mf\endcsname{\def\PYG@tc##1{\textcolor[rgb]{0.13,0.50,0.31}{##1}}}
\expandafter\def\csname PYG@tok@err\endcsname{\def\PYG@bc##1{\setlength{\fboxsep}{0pt}\fcolorbox[rgb]{1.00,0.00,0.00}{1,1,1}{\strut ##1}}}
\expandafter\def\csname PYG@tok@mb\endcsname{\def\PYG@tc##1{\textcolor[rgb]{0.13,0.50,0.31}{##1}}}
\expandafter\def\csname PYG@tok@ss\endcsname{\def\PYG@tc##1{\textcolor[rgb]{0.32,0.47,0.09}{##1}}}
\expandafter\def\csname PYG@tok@sr\endcsname{\def\PYG@tc##1{\textcolor[rgb]{0.14,0.33,0.53}{##1}}}
\expandafter\def\csname PYG@tok@mo\endcsname{\def\PYG@tc##1{\textcolor[rgb]{0.13,0.50,0.31}{##1}}}
\expandafter\def\csname PYG@tok@kd\endcsname{\let\PYG@bf=\textbf\def\PYG@tc##1{\textcolor[rgb]{0.00,0.44,0.13}{##1}}}
\expandafter\def\csname PYG@tok@mi\endcsname{\def\PYG@tc##1{\textcolor[rgb]{0.13,0.50,0.31}{##1}}}
\expandafter\def\csname PYG@tok@kn\endcsname{\let\PYG@bf=\textbf\def\PYG@tc##1{\textcolor[rgb]{0.00,0.44,0.13}{##1}}}
\expandafter\def\csname PYG@tok@o\endcsname{\def\PYG@tc##1{\textcolor[rgb]{0.40,0.40,0.40}{##1}}}
\expandafter\def\csname PYG@tok@kr\endcsname{\let\PYG@bf=\textbf\def\PYG@tc##1{\textcolor[rgb]{0.00,0.44,0.13}{##1}}}
\expandafter\def\csname PYG@tok@s\endcsname{\def\PYG@tc##1{\textcolor[rgb]{0.25,0.44,0.63}{##1}}}
\expandafter\def\csname PYG@tok@kp\endcsname{\def\PYG@tc##1{\textcolor[rgb]{0.00,0.44,0.13}{##1}}}
\expandafter\def\csname PYG@tok@w\endcsname{\def\PYG@tc##1{\textcolor[rgb]{0.73,0.73,0.73}{##1}}}
\expandafter\def\csname PYG@tok@kt\endcsname{\def\PYG@tc##1{\textcolor[rgb]{0.56,0.13,0.00}{##1}}}
\expandafter\def\csname PYG@tok@sc\endcsname{\def\PYG@tc##1{\textcolor[rgb]{0.25,0.44,0.63}{##1}}}
\expandafter\def\csname PYG@tok@sb\endcsname{\def\PYG@tc##1{\textcolor[rgb]{0.25,0.44,0.63}{##1}}}
\expandafter\def\csname PYG@tok@k\endcsname{\let\PYG@bf=\textbf\def\PYG@tc##1{\textcolor[rgb]{0.00,0.44,0.13}{##1}}}
\expandafter\def\csname PYG@tok@se\endcsname{\let\PYG@bf=\textbf\def\PYG@tc##1{\textcolor[rgb]{0.25,0.44,0.63}{##1}}}
\expandafter\def\csname PYG@tok@sd\endcsname{\let\PYG@it=\textit\def\PYG@tc##1{\textcolor[rgb]{0.25,0.44,0.63}{##1}}}

\def\PYGZbs{\char`\\}
\def\PYGZus{\char`\_}
\def\PYGZob{\char`\{}
\def\PYGZcb{\char`\}}
\def\PYGZca{\char`\^}
\def\PYGZam{\char`\&}
\def\PYGZlt{\char`\<}
\def\PYGZgt{\char`\>}
\def\PYGZsh{\char`\#}
\def\PYGZpc{\char`\%}
\def\PYGZdl{\char`\$}
\def\PYGZhy{\char`\-}
\def\PYGZsq{\char`\'}
\def\PYGZdq{\char`\"}
\def\PYGZti{\char`\~}
% for compatibility with earlier versions
\def\PYGZat{@}
\def\PYGZlb{[}
\def\PYGZrb{]}
\makeatother

\renewcommand\PYGZsq{\textquotesingle}

\begin{document}
\shorthandoff{"}
\maketitle
\tableofcontents
\phantomsection\label{index::doc}
\begin{notice}{danger}{Peligro:}
Work in progress:
\begin{itemize}
\item {} 
\textbf{{[}P{]}}: Pending (not even started)

\item {} 
\textbf{{[}S{]}}: Started (but not finished)

\end{itemize}

Not marked sections are finished.
\end{notice}




\chapter{Introducción a Ivozprovider}
\label{intro/index:documentacion-oficial-de-ivozprovider}\label{intro/index::doc}\label{intro/index:introduccion-a-ivozprovider}
Las siguientes secciones servirán como introducción general a IvozProvider:


\section{Sobre este documento}
\label{intro/about:sobre-este-documento}\label{intro/about::doc}
Este documento describe el proceso de instalación y de utilización de IvozProvider, la plataforma de telefonía multinivel orientada a operadores desarrollada por \href{http://irontec.com}{Irontec}.

Es el punto de partida para todo aquel interesado en esta solución, tanto desde el punto de vista técnico como desde el punto de vista de uso y consta de cuatro bloques:
\begin{itemize}
\item {} 
El primer bloque expone unos {\hyperref[index:concepts]{\sphinxcrossref{\DUrole{std,std-ref}{Conceptos iniciales}}}} que desglosan los elementos del producto y su función.

\item {} 
El siguiente bloque se describe el proceso de {\hyperref[index:installation]{\sphinxcrossref{\DUrole{std,std-ref}{Instalación y configuración mínima}}}} con la plataforma recién instalada, dejando de lado detalles que describirán en profundidad en el siguiente bloque.

\item {} 
El tercer bloque es el que profundiza en aspectos de {\hyperref[index:advanced]{\sphinxcrossref{\DUrole{std,std-ref}{Configuración avanzada}}}} como \textbf{tarificación, facturación, funcionalidades de PBX y todos los detalles} que en el bloque anterior se obviaron.

\item {} 
En el cuarto y último, se describen los mecanismos de {\hyperref[index:security]{\sphinxcrossref{\DUrole{std,std-ref}{Seguridad y mantenimiento}}}} que implementa la solución.

\end{itemize}


\section{Obtener ayuda}
\label{intro/getting_help:getting-help}\label{intro/getting_help::doc}\label{intro/getting_help:obtener-ayuda}
IvozProvider es un proyecto vivo y en continua evolución. Existen múltiples canales para obtener información o reportar errores:
\begin{itemize}
\item {} 
GitHub: \url{https://github.com/irontec/ivozprovider}

\item {} 
Listas de correo para usuarios: \href{mailto:users@lists-ivozprovider.irontec.com}{users@lists-ivozprovider.irontec.com}

\item {} 
Listas de correo para desarrolladores: \href{mailto:dev@lists-ivozprovider.irontec.com}{dev@lists-ivozprovider.irontec.com}

\item {} 
Dirección de correo: \href{mailto:vozip@irontec.com}{vozip@irontec.com}

\item {} 
Twitter: \href{https://twitter.com/irontec}{@irontec}

\item {} 
Canal IRC \href{https://webchat.freenode.net/?channels=ivozprovider}{\#ivozprovider}

\end{itemize}

No dudes en contactarnos cualquiera que sea el motivo :)


\section{¿Qué es IvozProvider?}
\label{intro/what_is_ivozprovider::doc}\label{intro/what_is_ivozprovider:que-es-ivozprovider}
IvozProvider es una solución de {\hyperref[intro/what_is_ivozprovider:voip]{\sphinxcrossref{\DUrole{std,std-ref}{telefonía IP}}}} {\hyperref[intro/what_is_ivozprovider:multilevel]{\sphinxcrossref{\DUrole{std,std-ref}{multinivel}}}} {\hyperref[intro/what_is_ivozprovider:operator\string-oriented]{\sphinxcrossref{\DUrole{std,std-ref}{orientada a operador}}}} {\hyperref[intro/what_is_ivozprovider:exposed]{\sphinxcrossref{\DUrole{std,std-ref}{expuesta a la red pública}}}}.


\subsection{Telefonía IP}
\label{intro/what_is_ivozprovider:voip}\label{intro/what_is_ivozprovider:telefonia-ip}
IvozProvider es compatible con los sistemas de telefonía que utilicen el protocolo \emph{Session Initiation Protocol}, \textbf{SIP}, descrito en el \href{https://tools.ietf.org/html/rfc3261}{RFC 3261} y todos los \href{https://www.packetizer.com/ipmc/sip/standards.html}{RFCs relacionados}, independientemente del fabricante.

Esto implica libertad total a la hora de elegir \emph{softphones}, \emph{hardphones} y el resto de elementos que interactúan con IvozProvider, sin ningún tipo de ataduras a ningún fabricante.

En lo relativo a los \textbf{protocolos de transporte} para transportar SIP, actualmente IvozProvider soporta:
\begin{itemize}
\item {} 
UDP

\item {} 
TCP

\item {} 
TLS

\item {} 
Websockets

\end{itemize}

Este último protocolo de transporte descrito en el \href{https://tools.ietf.org/html/rfc7118}{RFC 7118} permite la utilización de softphones desde la web, utilizando el estándar \href{https://webrtc.org/}{WebRTC} de los navegadores web para establecer comunicaciones en tiempo real \emph{peer-to-peer}.

En lo relativo a los \textbf{codecs de audio soportados}, la lista sería la siguiente:
\begin{itemize}
\item {} 
PCMA (\emph{alaw})

\item {} 
PCMU (\emph{ulaw})

\item {} 
G.729

\item {} 
\href{http://opus-codec.org/}{OPUS}

\end{itemize}


\subsection{Multinivel}
\label{intro/what_is_ivozprovider:multinivel}\label{intro/what_is_ivozprovider:multilevel}
El esquema de portales y diseño de IvozProvider permite que \textbf{múltiples actores cohabiten en una misma infraestructura}:

\noindent\sphinxincludegraphics{{operator_levels}.png}

En la sección {\hyperref[operation_roles/index:operation\string-roles]{\sphinxcrossref{\DUrole{std,std-ref}{Roles de la plataforma}}}} se describen los distintos roles en profundidad, pero se puede resumir en:
\begin{itemize}
\item {} 
\textbf{Nivel God}: instalador y mantenedor de la solución. Da acceso a n operadores de marca.

\item {} 
\textbf{Nivel Brand Operator}: Responsable de dar acceso, tarificar y facturar a n operadores de empresa.

\item {} 
\textbf{Nivel Company Operator}: Responsable de configurar el comportamiento de su centralita y dar de alta n usuarios finales.

\item {} 
\textbf{Nivel de Usuario}: Dispone de unas credenciales para sus terminales SIP y otra para el acceso a su panel web.

\end{itemize}

Cabe destacar que \textbf{cada uno} de estos niveles \textbf{dispone de un acceso web} que le permite realizar sus funciones. Los acceso web se pueden personalizar a nivel de:
\begin{itemize}
\item {} 
Temas y \emph{skins} con los colores corporativos.

\item {} 
Logo de la empresa.

\item {} 
URLs personalizadas con el dominio de la marca/empresa.

\end{itemize}


\subsection{Orientada a operador}
\label{intro/what_is_ivozprovider:operator-oriented}\label{intro/what_is_ivozprovider:orientada-a-operador}
IvozProvider es una solución de telefonía \textbf{diseñada con el escalado horizontal en mente}, lo que la permite adecuarse a \textbf{altos volúmenes de tráfico y de usuarios} sin más que adaptar el número de máquinas y los recursos de éstas.

Estas son las ideas principales que lo hacen un producto orientado a operador:
\begin{itemize}
\item {} 
A pesar de que todas las piezas pueden correr en una misma máquina, lo que facilita las pruebas iniciales, cada elemento de IvozProvider se puede separar del resto y correr en su propio hardware.

\item {} 
Una \textbf{instalación distribuida} permite asignar máquinas con recursos adecuados a cada tarea, pero también posibilita:
\begin{itemize}
\item {} 
Separación geográfica de los elementos para garantizar una disponibilidad a prueba de fallo de CPD.

\item {} 
Instalación de elementos clave cerca de los usuarios finales, para minimizar latencias en sus comunicaciones.

\item {} 
Escalado horizontal de los elementos clave para dar servicios a cientos de miles de llamadas concurrentes.

\end{itemize}

\end{itemize}

Los elementos que limitan la capacidad de servicio de las soluciones VoIP suelen ser:
\begin{itemize}
\item {} 
Gestión del audio de las llamadas establecidas.

\item {} 
Gestión de las lógicas programadas por cada administador de empresa (IVRs, salas de conferencias, filtros de horario, etc.)

\item {} 
Bases de datos de almacenamiento de configuraciones y registros.

\end{itemize}

IvozProvider ha sido diseñado con la idea de mantener el \textbf{escalado horizontal} de cada uno de estos elementos, \textbf{para así poder llegar a poder gestionar cientos de miles de llamadas concurrentes} y, lo que es más importante, poder \textbf{adaptar los recursos de la plataforma al nivel de servicio que se espera en cada momento}:
\begin{itemize}
\item {} 
Los \textbf{Media-relays} se encargan de reenviar las tramas de audio de las sesiones establecidas:
\begin{itemize}
\item {} 
Se pueden poner tantos media-relays como sean necesarios.

\item {} 
Se pueden crear grupos de media-relays y hacer una asignación estática a las empresas deseadas.

\item {} 
Se pueden poner los media-relays cerca de los usuarios finales, para evitar latencias en las llamadas.

\end{itemize}

\item {} 
Los \textbf{Servidores de Aplicación} se encargan de la fase previa de toda llamada: hacer que siga la lógica programada. Este rol:
\begin{itemize}
\item {} 
Se puede escalar horizontalmente: que los Servidores de Aplicación empiezan a estar saturados, se instalan más y se añaden al pool.

\item {} 
Una llamada acaba en el Servidor de Aplicación que menos cargado esté en cada momento.

\item {} 
En la configuración por defecto no existe asignación estática \footnote[*]{\sphinxAtStartFootnote%
El administrador global puede asignar Servidores de Aplicación estáticamente a empresas, pero esta funcionalidad está pensada como herramienta de \emph{troubleshooting} puntual.
} que envíe las llamadas de una empresa a un Servidor de Aplicación concreto, de modo que la caída de cualquier Servidor de Aplicación no es crítica: el sistema dejará de contar con él a la hora de distribuir las llamadas.

\end{itemize}

\end{itemize}


\subsection{Expuesta a la red pública}
\label{intro/what_is_ivozprovider:expuesta-a-la-red-publica}\label{intro/what_is_ivozprovider:exposed}
Tal y como se verá en el proceso de instalación, \textbf{IvozProvider está diseñado para servir a usuarios directamente desde Internet}. Aunque pueda utilizarse en entornos locales, IvozProvider se ha diseñado para disponer de direcciones IPs públicas para dar servicio sin necesidad de túneles VPN o IPsec que te conecten con la infraestructura.

Cabe destacar:
\begin{itemize}
\item {} 
Solo los elementos imprescindibles están expuestos a Internet.

\item {} 
Los accesos desde países de dudosa confianza se cortan en el firewall incorporado.

\item {} 
Se puede filtrar el acceso desde IPs/redes autorizadas para evitar fraudes.

\item {} 
Existe un mecanismo anti-flood para evitar grandes consumos en poco tiempo.

\item {} 
Existe un mecanismo de control de llamadas concurrentes por empresa.

\item {} 
IvozProvider soporta la conexión desde terminales tras \href{https://es.wikipedia.org/wiki/Traducci\%C3\%B3n\_de\_direcciones\_de\_red}{NAT}.

\item {} 
IvozProvider se encarga de mantener activas dichas ventanas de NAT con mecanismos de \emph{nat-piercing}.

\end{itemize}


\section{¿Qué hay dentro de IvozProvider?}
\label{intro/what_is_inside:que-hay-dentro-de-ivozprovider}\label{intro/what_is_inside::doc}
IvozProvider incluye proyectos de \href{https://www.gnu.org/philosophy/free-sw.es.html}{Software Libre} de reconocido prestigio y larga trayectoria para cumplir las tareas de diversa índole que necesita la plataforma.

Nada como una imagen para ilustrar los distintos proyectos que integra IvozProvider:

\noindent{\hspace*{\fill}\sphinxincludegraphics{{ivozprovider_logos}.png}\hspace*{\fill}}

\begin{notice}{note}{Nota:}
Nuestro más sincero agradecimiento a los desarrolladores de todos estos proyectos, así como a las comunidades de usuarios.
\end{notice}

La función que cumple cada una de estas piezas se detallará con mayor profundidad en el bloque titulado {\hyperref[architecture/index:architecture]{\sphinxcrossref{\DUrole{std,std-ref}{{[}P{]} Arquitectura general de la plataforma}}}}.


\section{¿Quién debería de usar IvozProvider?}
\label{intro/use_cases::doc}\label{intro/use_cases:quien-deberia-de-usar-ivozprovider}
IvozProvider es un buen candidato para todo aquel que esté interesado en disponer de una plataforma de telefonía para poder dar servicios a \textbf{cientos de miles de llamadas de forma concurrente}.

Las grandes fortalezas de IvozProvider pueden servir para saber si IvozProvider encaja en tu caso concreto:
\begin{itemize}
\item {} 
VoIP: SIP.

\item {} 
Multinivel, multimarca.

\item {} 
Escalado horizontal.

\item {} 
PseudoSBC: abierto a Internet.

\item {} 
Funcionalidades de PBX.

\end{itemize}

El proceso de instalación es tan simple, que la mejor forma de saber si IvozProvider es para ti, ¡es probarlo!


\chapter{{[}P{]} Arquitectura general de la plataforma}
\label{architecture/index:p-arquitectura-general-de-la-plataforma}\label{architecture/index::doc}\label{architecture/index:architecture}

\section{Esquema general}
\label{architecture/index:esquema-general}
Este esquema refleja la arquitectura global de la solución IvozProvider, con todos los elementos que la componen.

\noindent\sphinxincludegraphics{{ivozprovider_flows}.png}

Este es otro esquema más conceptual:

\noindent\sphinxincludegraphics{{ivozprovider_esquema_conceptual}.png}


\section{Flujo señalización SIP}
\label{architecture/index:flujo-senalizacion-sip}\label{architecture/index:signallingflow}

\section{Flujo audio RTP}
\label{architecture/index:flujo-audio-rtp}\label{architecture/index:audioflow}

\section{Tráfico HTTPS}
\label{architecture/index:trafico-https}

\section{Elementos adicionales}
\label{architecture/index:elementos-adicionales}

\section{Elementos auxiliares}
\label{architecture/index:elementos-auxiliares}

\chapter{Roles de la plataforma}
\label{operation_roles/index:operation-roles}\label{operation_roles/index:roles-de-la-plataforma}\label{operation_roles/index::doc}
Ivozprovider es una solución orientada a operador con múltiples niveles de usuarios.

La siguiente imagen ilustra los niveles que existen, así como la relación entre ellos:

\noindent\sphinxincludegraphics{{operator_levels1}.png}

En esta sección se explicará cada uno de estos roles, describiendo sus responsabilidades y sus funciones más importantes.


\section{Rol de administrador global}
\label{operation_roles/index:rol-de-administrador-global}
El rol de administrador global (operador en la imagen) lo desempeña habitualmente instalador de IvozProvider.

Tiene visibilidad total de todos los aspectos de la plataforma y suele ser el encargado del mantenimiento de la misma.

Su \textbf{función más importante es crear Marcas} y hacer todo lo necesario para que dispongan de la autonomía necesaria para usar la plataforma:
\begin{itemize}
\item {} 
Configurar sus accesos web.

\item {} 
Configurar el aspecto de su portal de administración de marca: tema, colores, etc.

\end{itemize}

Aparte de esta función principal, su visibilidad global y acceso total le hacen responsable de:
\begin{itemize}
\item {} 
Monitorizar la plataforma para que esté siempre UP \& RUNNING.

\item {} 
Analizar los logs de la plataforma en busca de posibles errores.

\item {} 
Afinar los mecanismos de seguridad para evitar ataques externos.

\item {} 
Obtener estadísticas globales de calidad de llamada.

\item {} 
Ir aumentando los recursos de la plataforma a medida que se vaya necesitando:
\begin{itemize}
\item {} 
Aumentando los recursos de la instalación standalone.

\item {} 
Migrando, llegado el momento, a una instalación distribuida con múltiples AS-es, media relays, etc.

\end{itemize}

\end{itemize}

En resumen, \textbf{es el único rol que no tiene límites dentro de la plataforma}, de ahí la denominación \emph{God} que se utilizará en múltiples lugares de esta documentación.

\begin{notice}{important}{Importante:}
\textbf{Este rol se encarga de mantener la plataforma}, adaptándola a las necesidades de cada momento. Su rol \textbf{no tiene ningún tipo de límite y} es el que \textbf{da acceso a} los \emph{n} \textbf{operadores de marca}.
\end{notice}


\section{Rol de operador de marca}
\label{operation_roles/index:rol-de-operador-de-marca}
El operador de marca utiliza la plataforma con un acceso y una visibilidad menor que el rol anterior. En concreto, el operador global le facilita una URL y unas credenciales y ese portal web de operador de Marca es su único interfaz con IvozProvider.

No obstante, este portal le permite desempeñar su función más importante, que es \textbf{crear empresas y configurar todo lo necesario para que éstas desempeñen su función}.

Dado que el operador de marca es el encargado de facturar a sus empresas y de hacer que sus llamadas salgan al exterior, también tiene que gestionar:
\begin{itemize}
\item {} 
Contratos de Peering con otros operadores IP para conectar con la PSTN.

\item {} 
Incluir en los datos de la empresa toda información necesaria para las facturas.

\item {} 
Planes de Precio que ofertarán a sus empresas, qué determinarán cuánto pagan por cada tipo de llamada.

\item {} 
Configurar por dónde sale cada tipo de llamada de cada empresa, en función del destino.

\item {} 
Generar las facturas en cada período de facturación y emitarlas al cliente.

\end{itemize}

Como se puede ver, las tareas del operador de marca poco tienen que ver con las del operador global, pero son vitales para que los usuarios finales puedan hacer uso de las funcionalidades de IvozProvider.
\phantomsection\label{operation_roles/index:brand-responsibilities}
\begin{notice}{important}{Importante:}
\textbf{En resumen}, los operadores de marca \textbf{dan acceso a} los administradores de las \textbf{empresas} a las que den servicio \textbf{y configuran la plataforma para poder enrutar, tarificar y facturar sus llamadas}.
\end{notice}


\section{Rol de administrador de empresa}
\label{operation_roles/index:rol-de-administrador-de-empresa}
El administrador de empresa dispone del acceso web que le proveé el administrador de marca.

Desde su perspectiva, dispone de una centralita virtual en la nube que tiene que configurar para que la utilicen sus usuarios.

Para ello, tendrá que:
\begin{itemize}
\item {} 
Dar de alta terminales, extensiones y usuarios.

\item {} 
Configurar el tratamiento de los DDIs de entrada, para que se comporten como quieran:
\begin{itemize}
\item {} 
Directos a usuario.

\item {} 
IVRs.

\item {} 
Grupos de Salto.

\item {} 
Faxes.

\end{itemize}

\item {} 
Dar acceso a los usuarios finales a su portal web, para que configuren a su gusto aspectos como:
\begin{itemize}
\item {} 
Desvíos

\item {} 
No molestar

\item {} 
Llamada en espera

\end{itemize}

\end{itemize}

\begin{notice}{important}{Importante:}
En resumen, los administradores de empresa son los responsables de \textbf{configurar su sistema de telefonía a su gusto y de utilizar todas las funcionalidades que proveé IvozProvider}.
\end{notice}


\section{Rol de usuario final}
\label{operation_roles/index:rol-de-usuario-final}
El usuario final dispondrá de dos credenciales, ambas provistas por su administrador de empresa:
\begin{itemize}
\item {} 
Credenciales de acceso web al portal de usuario.

\item {} 
Credenciales SIP para registrar su terminal (o terminales) contra IvozProvider.

\end{itemize}

Desde el portal de usuario podrán ver sus registros de llamadas y configurar aspectos como:
\begin{itemize}
\item {} 
Desvíos

\item {} 
No molestar

\item {} 
Llamada en espera

\end{itemize}

Por otra parte, las credenciales SIP le permitirán configurar su terminal (o terminales) para poder emitir y recibir llamadas.

\begin{notice}{note}{Nota:}
Unas mismas credenciales se pueden utilizar desde múltiples dispositivos, dando lugar a lo que se conoce como \emph{parallel-forking}: si llaman al usuario, sonarán todos sus dispositivos activos y podrá contestar la llamada desde cualquiera de ellos.
\end{notice}

\begin{notice}{important}{Importante:}
Los usuarios son los que utilizan y disfrutan todas las funcionalidades de IvozProvider.
\end{notice}


\chapter{{[}S{]} Instalación inicial}
\label{installation/index:s-instalacion-inicial}\label{installation/index::doc}

\section{Tipos de instalación}
\label{installation/install_types::doc}\label{installation/install_types:tipos-de-instalacion}

\subsection{Instalación standalone}
\label{installation/install_types:instalacion-standalone}\label{installation/install_types:id1}

\subsection{Instalación distribuida}
\label{installation/install_types:instalacion-distribuida}\label{installation/install_types:id2}

\section{Requisitos mínimos}
\label{installation/requirements:requisitos-minimos}\label{installation/requirements::doc}

\subsection{Requerimientos del sistema}
\label{installation/requirements:requerimientos-del-sistema}

\subsection{Configuración de red}
\label{installation/requirements:configuracion-de-red}

\subsection{Configuración DNS}
\label{installation/requirements:configuracion-dns}

\section{Instalación por paquetes Debian}
\label{installation/debian_install::doc}\label{installation/debian_install:instalacion-por-paquetes-debian}
IvozProvider está diseñado para instalarse y actualizarse mediante paquetes Debian. En concreto, la release actual esta pensada para funcionar sobre \href{https://www.debian.org/releases/jessie}{Debian Jessie 8}.

Se recomienda emplear las \href{https://www.debian.org/releases/jessie/installmanual}{guias oficiales de instalación} para obtener un sistema base mínimo, ya que toda dependencia que necesite posteriormente será instalada automaticamente.

Tanto si deseas realizar una {\hyperref[installation/install_types:instalacion\string-standalone]{\sphinxcrossref{\DUrole{std,std-ref}{Instalación standalone}}}} o una {\hyperref[installation/install_types:instalacion\string-distribuida]{\sphinxcrossref{\DUrole{std,std-ref}{Instalación distribuida}}}}, es preciso configurar los repositorios de paquetes debian de Irontec.


\subsection{Configurar repositorios APT}
\label{installation/debian_install:configurar-repositorios-apt}
Actualmente se emplean dos repositorios diferentes tanto para la última release de IvozProvider (oasis) como para la de Klear (chloe)

\begin{Verbatim}[commandchars=\\\{\}]
\PYG{g+go}{cd /etc/apt/sources.list.d}
\PYG{g+go}{echo deb http://packages.irontec.com/debian oasis main extra \PYGZgt{} ivozprovider.list}
\PYG{g+go}{echo deb http://packages.irontec.com/debian chloe main \PYGZgt{} klear.list}
\end{Verbatim}

Añadimos la clave publica del repositorio:

\begin{Verbatim}[commandchars=\\\{\}]
\PYG{g+go}{wget http://packages.irontec.com/public.key \PYGZhy{}q \PYGZhy{}O \PYGZhy{} \textbar{} apt\PYGZhy{}key add \PYGZhy{}}
\end{Verbatim}


\subsection{Instalar el paquete del rol}
\label{installation/debian_install:instalar-el-paquete-del-rol}
Una vez configurados los repositorios será preciso seleccionar el paquete acorde al perfil que queramos instalar:
\begin{itemize}
\item {} \begin{description}
\item[{Para una {\hyperref[installation/install_types:instalacion\string-standalone]{\sphinxcrossref{\DUrole{std,std-ref}{Instalación standalone}}}}:}] \leavevmode\begin{itemize}
\item {} 
ivozprovider

\end{itemize}

\end{description}

\item {} \begin{description}
\item[{Para una {\hyperref[installation/install_types:instalacion\string-distribuida]{\sphinxcrossref{\DUrole{std,std-ref}{Instalación distribuida}}}}: uno de los paquetes en función rol se desee que desempeñe la máquina}] \leavevmode\begin{itemize}
\item {} 
ivozprovider-profile-data

\item {} 
ivozprovider-profile-proxy

\item {} 
ivozprovider-profile-portal

\item {} 
ivozprovider-profile-as

\end{itemize}

\end{description}

\end{itemize}

\begin{Verbatim}[commandchars=\\\{\}]
\PYG{g+go}{apt\PYGZhy{}get update}
\PYG{g+go}{apt\PYGZhy{}get install ivozprovider}
\end{Verbatim}


\subsection{Completar instalación}
\label{installation/debian_install:completar-instalacion}
Las instalaciones distribuidas requieren multiples configuraciones en funcion del rol que se haya instalado. Consulte \href{http://google.com}{completar la instalción de un rol} para más información.

Las instalaciones standalone cuentan con un menú que ayuda a configurar los datos básicos de los servicios empleados en IvozProvider. Puesto que todos los servicios se ejecutan en la misma máquina, muchos de los procesos vienen configurados automáticamente con los valores por defecto.

El menú permite, entre otros:
\begin{itemize}
\item {} 
Configurar las IPs públicas de los proxies SIP

\item {} 
El lenguaje por defecto que empleará la plataforma

\item {} 
Las contraseñas para acceder a las bases de datos

\end{itemize}

Es posible cambiar cualquiera de estos valores una vez instalado IvozProvider volviendo a ejecutar:

\begin{Verbatim}[commandchars=\\\{\}]
\PYG{g+go}{dpkg\PYGZhy{}reconfigure ivozprovider}
\end{Verbatim}


\section{Instalación desde DockerHub}
\label{installation/docker_install:instalacion-desde-dockerhub}\label{installation/docker_install::doc}

\chapter{Realizar llamadas internas}
\label{internal_calls/index::doc}\label{internal_calls/index:realizar-llamadas-internas}
El objetivo de este bloque será configurar IvozProvider para realizar llamadas internas, partiendo de la instalación base descrita en la sección anterior.

Para conseguir que Alice llame a Bob, tendremos que realizar labores de 3 niveles descritos en {\hyperref[operation_roles/index:operation\string-roles]{\sphinxcrossref{\DUrole{std,std-ref}{Roles de la plataforma}}}}, de ahí la organización del siguiente índice en esos 3 bloques:


\section{Bloque `Gestión General'}
\label{internal_calls/god_portal::doc}\label{internal_calls/god_portal:bloque-gestion-general}
\begin{notice}{important}{Importante:}
Cualquiera de las 2 IPs públicas configuradas en la instalación servirá para acceder al panel web. Las credenciales por defecto son \emph{admin / changeme}.
\end{notice}

En esta sección haremos referencia a todo lo relativo al rol operador global, configurable en el bloque \textbf{Gestión general} del panel web (solo visible para God):

\noindent{\hspace*{\fill}\sphinxincludegraphics{{bloque_god}.png}\hspace*{\fill}}


\subsection{Configuración personalizada en la instalación}
\label{internal_calls/god_portal:configuracion-personalizada-en-la-instalacion}
En el proceso de instalación se pregunta al administrador dos direcciones IP, con el fin de arrancar los siguientes 2 procesos:


\subsubsection{Proxy de Usuarios}
\label{internal_calls/god_portal:proxyusers}\label{internal_calls/god_portal:proxy-de-usuarios}
Es el proxy SIP expuesto al mundo exterior al que se registran los terminales de los usuarios.

El valor mostrado en la sección \textbf{Proxy de Usuarios} reflejará la IP introducida en el proceso de instalación.

\noindent\sphinxincludegraphics{{proxyusers}.png}


\subsubsection{Proxy de Salida}
\label{internal_calls/god_portal:proxy-de-salida}
Es el proxy SIP expuesto al mundo exterior al que hablarán los Operadores IP con los que el operador de marca decida hacer \emph{peering}.

El valor mostrado en la sección \textbf{Proxy de Salida} reflejará la IP introducida en el proceso de instalación.

\noindent\sphinxincludegraphics{{proxytrunks}.png}

\begin{notice}{note}{Nota:}
Solo se explicita la dirección IP, ya que el puerto siempre será 5060 (5061 para SIP sobre TLS).
\end{notice}

\begin{notice}{danger}{Peligro:}
Estos 2 valores pueden editarse desde la web, pero siempre tienen que tener la dirección IP a la que escuchan dichos procesos.
\end{notice}


\subsection{Configuración global estándar}
\label{internal_calls/god_portal:configuracion-global-estandar}
El proceso de instalación incluye otros valores globales que son iguales en toda instalación de IvozProvider (standalone) y que también se pueden ver desde la interfaz web.


\subsubsection{Servidores de aplicación}
\label{internal_calls/god_portal:servidores-de-aplicacion}
En la sección \textbf{Servidores de Aplicación} se listan las direcciones IP donde escuchan los distintos Asterisk que componen la solución que, tal y como se ha mencionado, escalan horizontalmente para adaptarse a la carga de la plataforma.

A diferencia de los Proxies, estos Asterisk no están expuestos al exterior, por lo que en una instalación standalone solo habrá uno y escuchará en 127.0.0.1.

\noindent\sphinxincludegraphics{{app_servers}.png}

\begin{notice}{note}{Nota:}
El puerto en el que escuchan no se recoge en el campo, ya que siempre será 6060 (UDP).
\end{notice}

\begin{notice}{important}{Importante:}
Desde el momento en el que se añade otro Servidor de Aplicación, se intentará contar con él a la hora de repartir la carga. Si éste no responde, se desactivará automáticamente.
\end{notice}


\subsubsection{Servidores de Media}
\label{internal_calls/god_portal:servidores-de-media}
Los media-relays son los que mueven el tráfico RTP en una llamada establecida y, al igual que ocurre con los Servidores de Aplicación, permiten un escalado horizontal para adaptarse a la carga de la plataforma.

Los media-relays se organizan en grupos con el fin de poder asignar un grupo concreto a una empresa concreta. Cada elemento del grupo tiene una \textbf{métrica} que permite repartos de carga desiguales dentro de un mismo grupo (e.g. media-relay1 métrica 1; media-relay2 métrica 2: media-relay2 gestionará el audio del doble de llamadas que media-relay1).

\begin{notice}{hint}{Consejo:}
La asignación de grupos de media-relays concretos a empresas concretas permite asignar recursos estáticos a empresas que requieren tener garantizado unos recursos concretos. Pero, lo más útil de este tipo de configuración es que estos \textbf{grupos de media-relays pueden estar en ubicaciones geográficas cercanas al emplazamiento de la empresa} (y lejanas al resto de la plataforma) para \textbf{reducir las latencias} en sus conversaciones.
\end{notice}

En una instalación standalone, no obstante, solo existe un grupo de media-relays:

\noindent\sphinxincludegraphics{{media_relay_groups}.png}

Por defecto, este grupo solo contiene un media-relay:

\noindent\sphinxincludegraphics{{media_relays}.png}

\begin{notice}{note}{Nota:}
La dirección que aparece es la dirección del socket de control, no la dirección que se acaba incluyendo en los SDPs de negociación de sesión. Por defecto, este único media-relay utiliza la misma IP que el Proxy de Usuarios.
\end{notice}


\subsubsection{Dominios SIP}
\label{internal_calls/god_portal:dominios-sip}\label{internal_calls/god_portal:god-sipdomains}
En la sección \textbf{Dominios} se muestran los dominios SIP que apuntan a las 2 IPs públicas:
\begin{itemize}
\item {} 
IP de Proxy de Usuarios

\item {} 
IP de Proxy de Salida

\end{itemize}

Tras una instalación inicial existen 2 dominios, uno para queda una de esas 2 IPs:

\noindent\sphinxincludegraphics{{domain_list_local}.png}

Estos dominios se utilizan internamente y el servidor de DNS incorporado en la solución los resuelve a las IPs concretas.

\begin{notice}{attention}{Atención:}
Tal y como se verá en la sección {\hyperref[internal_calls/brand_portal:domain\string-per\string-company]{\sphinxcrossref{\DUrole{std,std-ref}{Dominio SIP de la compañía}}}}, cada compañía necesitará un DNS que apunte a la IP del Proxy de Usuarios. Una vez configurado, el dominio aparecerá en esta sección para que el operador global sepa los dominios configurados para cada empresa de un vistazo.
\end{notice}


\subsection{Emular la marca Demo}
\label{internal_calls/god_portal:emular-la-marca-demo}
Tras la instalación inicial, la plataforma incluye una marca pre-creada llamada DemoBrand, que es la que utilizaremos para el fin que nos ocupa: tener 2 teléfonos registrados y que se puedan llamar entre sí.

Antes de pasar a la siguiente sección, es importante entender el concepto de \textbf{Emular una marca}:
\begin{itemize}
\item {} 
Como operador global, tienes acceso al bloque \textbf{Gestión general}, que solo ve \emph{God}.

\item {} 
Aparte de ese bloque, también ves los bloques \textbf{Configuración de marca} y \textbf{Configuración de empresa} que tienen este aspecto:

\end{itemize}

\noindent{\hspace*{\fill}\sphinxincludegraphics{{emular_marca_prev}.png}\hspace*{\fill}}
\begin{itemize}
\item {} 
Atención especial al siguiente botón:

\end{itemize}

\noindent{\hspace*{\fill}\sphinxincludegraphics{{emular_marca}.png}\hspace*{\fill}}
\begin{itemize}
\item {} 
Una vez pulsado, muestra una ventana flotante tal que:

\end{itemize}

\noindent{\hspace*{\fill}\sphinxincludegraphics{{emular_marca2}.png}\hspace*{\fill}}
\begin{itemize}
\item {} 
Al seleccionar la marca DemoBrand, el icono cambia y muestra la marca que se está emulando:

\end{itemize}

\noindent{\hspace*{\fill}\sphinxincludegraphics{{emular_marca3}.png}\hspace*{\fill}}
\begin{itemize}
\item {} 
La parte superior derecha de la página también muestra la marca que se está emulando:

\end{itemize}

\noindent{\hspace*{\fill}\sphinxincludegraphics{{emular_marca4}.png}\hspace*{\fill}}


\subsubsection{¿Qué implica esta emulación?}
\label{internal_calls/god_portal:que-implica-esta-emulacion}
Que \textbf{todo lo que se ve en el bloque `Configuración de marca' es relativo a esa marca} y es \emph{exactamente} lo mismo que lo que ve el operador de marca cuando entra con sus credenciales de acceso.

\begin{notice}{tip}{Truco:}
Decir exactamente es mucho decir, ya que el operador global ve campos en ciertas secciones del bloque \textbf{Configuración de marca} que el operador de marca no ve. e.g. Al editar una empresa \emph{God} ve `Servidores de media' y `AS', que el operador de marca no ve.
\end{notice}


\section{Bloque `Configuración de Marca'}
\label{internal_calls/brand_portal:bloque-configuracion-de-marca}\label{internal_calls/brand_portal::doc}
Para conseguir que esta DemoBrand tenga una compañía con 2 usuarios que se puedan llamar entre sí, vamos a tener que hacer muy poco en este bloque.

De hecho, al acceder a la sección \textbf{Empresas}, vemos que ya existe una compañía \emph{DemoCompany} que podremos utilizar para cumplir nuestro ansiado objetivo :)

\noindent\sphinxincludegraphics{{company_list}.png}

Solo le falta una cosa a esta empresa, marcado con \textbf{EDIT} en la captura anterior.


\subsection{Dominio SIP de la compañía}
\label{internal_calls/brand_portal:dominio-sip-de-la-compania}\label{internal_calls/brand_portal:domain-per-company}
Tal y como se introdujo en la sección anterior, es \textbf{imprescindible} que cada empresa tenga un dominio público que resuelva a la IP configurada para el {\hyperref[internal_calls/god_portal:proxyusers]{\sphinxcrossref{\DUrole{std,std-ref}{Proxy de Usuarios}}}}.

\begin{notice}{note}{Nota:}
El registro DNS puede ser de tipo A (soportado por todos los hardphones/softphones) o del tipo NAPTR+SRV.
\end{notice}

Una vez configurado el dominio (por medio de procedimientos que escapan al objetivo de este documento), bastará con escribir el parámetro en el campo adecuado de nuestra empresa:

\noindent\sphinxincludegraphics{{set_domain}.png}

Una vez guardada la empresa, este dominio aparecerá en la sección descrita {\hyperref[internal_calls/god_portal:god\string-sipdomains]{\sphinxcrossref{\DUrole{std,std-ref}{en la sección anterior}}}}:

\noindent\sphinxincludegraphics{{domain_list}.png}

\begin{notice}{attention}{Atención:}
Es fundamental entender este bloque, sin un registro DNS correctamente configurado apuntando a la IP del Proxy de Usuarios, ¡fracasaremos en nuestro objetivo!
\end{notice}

Esta es una buena señal para el dominio que acabamos de configurar, siempre y cuando en lugar de 10.10.3.10 aparezca la IP pública configurada en {\hyperref[internal_calls/god_portal:proxyusers]{\sphinxcrossref{\DUrole{std,std-ref}{Proxy de Usuarios}}}}.

\noindent\sphinxincludegraphics{{dominio_bien_configurado}.png}

\begin{notice}{danger}{Peligro:}
¿Se ha insistido suficiente en que sin un registro DNS correctamente configurado apuntando a la IP del Proxy de Usuarios no funcionará nada?
\end{notice}


\subsubsection{No tengo tiempo para crear registros DNS}
\label{internal_calls/brand_portal:dnshack}\label{internal_calls/brand_portal:no-tengo-tiempo-para-crear-registros-dns}
Todo lo contado hasta este punto es verídico: a medida que vayamos creando marcas y éstas vayan creando empresas, cada una de ellas necesitará un registro DNS.

Pero la primera empresa de la plataforma es especial y puede apoderarse de la IP del Proxy de Usuarios y usarla como si de un DNS se tratara:

\noindent\sphinxincludegraphics{{fake_domain}.png}

A pesar de no ser un dominio, al estar usándose como tal, aparecerá en la sección de \textbf{Dominios}:

\noindent\sphinxincludegraphics{{fake_domain2}.png}

\begin{notice}{tip}{Truco:}
Es importante entender que este truco solo es válido para la primera empresa de la plataforma ;)
\end{notice}


\subsection{Emular la empresa Demo}
\label{internal_calls/brand_portal:emular-la-empresa-demo}\label{internal_calls/brand_portal:emulate-company}
El proceso de emulación de empresa es idéntico al de emulación de marca, con la diferencia de que filtra el bloque `Configuración de Empresa' en lugar del bloque `Configuración de Marca'.

\noindent{\hspace*{\fill}\sphinxincludegraphics{{emulate_company}.png}\hspace*{\fill}}

\noindent{\hspace*{\fill}\sphinxincludegraphics{{emulate_company2}.png}\hspace*{\fill}}

Una vez emulada la empresa, la parte superior derecha de la pantalla mostrará que vamos por el buen camino :)

\noindent{\hspace*{\fill}\sphinxincludegraphics{{emular_empresa}.png}\hspace*{\fill}}


\section{Bloque `Configuración de Empresa'}
\label{internal_calls/company_portal:bloque-configuracion-de-empresa}\label{internal_calls/company_portal::doc}
Estamos cerca de hacer nuestra primera llamada en nuestro flamente nuevo IvozProvider, solo queda crear 6 cosas dentro de nuestra empresa DemoCompany:
\begin{itemize}
\item {} 
2 terminales

\item {} 
2 extensiones

\item {} 
2 usuarios

\end{itemize}


\subsection{Terminales}
\label{internal_calls/company_portal:terminales}
Vamos a la sección terminales y... voilà! Ya tenemos 2 terminales pre-creados:

\noindent\sphinxincludegraphics{{terminals}.png}


\subsection{Extensiones}
\label{internal_calls/company_portal:extensiones}
Seguimos por tanto en la sección de extensiones, pero se nos han adelantado y ya tenemos 2 extensiones para nuestro uso:

\noindent\sphinxincludegraphics{{extensions}.png}

Nada por hacer en esta sección tampoco, ¡vamos a la última!


\subsection{Usuarios}
\label{internal_calls/company_portal:usuarios}
Como era de esperar, también tenemos 2 usuarios:

\noindent\sphinxincludegraphics{{users}.png}

Llegados a este punto y sin necesidad de tocar nada en este bloque, ya tenemos todo listo para hacer una llamada de Alice a Bob.


\section{Configurar terminales SIP}
\label{internal_calls/configure_sipuacs::doc}\label{internal_calls/configure_sipuacs:configurar-terminales-sip}
Lo único que nos falta es disponer de 2 terminales SIP (hardphone, softphone, Android/IOS APP) y configurarlos como sigue:

\textbf{ALICE}
\begin{itemize}
\item {} 
\textbf{Usuario}: alice

\item {} 
\textbf{Contraseña}: alice

\item {} 
\textbf{Dominio}: users.democompany.com (o la IP si hemos hecho {\hyperref[internal_calls/brand_portal:dnshack]{\sphinxcrossref{\DUrole{std,std-ref}{el truco}}}})

\end{itemize}

\textbf{BOB}
\begin{itemize}
\item {} 
\textbf{Usuario}: bob

\item {} 
\textbf{Contraseña}: bob

\item {} 
\textbf{Dominio}: users.democompany.com (o la IP si hemos hecho {\hyperref[internal_calls/brand_portal:dnshack]{\sphinxcrossref{\DUrole{std,std-ref}{el truco}}}})

\end{itemize}

\begin{notice}{tip}{Truco:}
Es posible que el usuario y el dominio se nos pida junto, tendremos que introducir \href{mailto:alice@users.democompany.com}{alice@users.democompany.com} y \href{mailto:bob@users.democompany.com}{bob@users.democompany.com}, respectivamente (o con la IP si hemos hecho {\hyperref[internal_calls/brand_portal:dnshack]{\sphinxcrossref{\DUrole{std,std-ref}{el truco}}}}).
\end{notice}

Tras configurar los terminales, Alice debería de poder llamar a Bob sin más que marcar 102 en su terminal.


\chapter{Recibir llamadas externas}
\label{external_incoming_calls/index:recibir-llamadas-externas}\label{external_incoming_calls/index::doc}
El objetivo de este bloque será configurar IvozProvider para recibir llamadas externas.

Para ello, seguiremos los siguientes pasos:


\section{Transformaciones numéricas}
\label{external_incoming_calls/numeric_transformations:transformaciones-numericas}\label{external_incoming_calls/numeric_transformations::doc}\label{external_incoming_calls/numeric_transformations:numeric-transformations}

\subsection{Concepto}
\label{external_incoming_calls/numeric_transformations:concepto}
\textbf{IvozProvider} está diseñado con la intención de \textbf{poder dar servicio en cualquier lugar del planeta}, no solamente en el país originario de la solución.

Un concepto muy importante para conseguir este objetivo es el de las transformaciones númericas, que consisten en la adaptación de los distintos sistemas de numeración de los países del mundo definidos en \href{https://www.itu.int/rec/T-REC-E.164/es}{E.164} a un formato neutro.

La sección que le permite configurar al operador de marca todo lo relativo a \textbf{Transformaciones numéricas} es:

\noindent{\hspace*{\fill}\sphinxincludegraphics{{numeric_transformations_section}.png}\hspace*{\fill}}

En concreto, se distinguen 2 casos:


\subsubsection{Adaptación en entrada}
\label{external_incoming_calls/numeric_transformations:adaptacion-en-entrada}
Cuando una llamada entra a IvozProvider procedente de un operador con el que tenemos un acuerdo de \emph{peering}, hay que adaptar las numeraciones que hacen referencia a:
\begin{itemize}
\item {} 
Origen de la llamada.

\item {} 
Destino de la llamada.

\end{itemize}

En función del país del operador indicará los números internacionales de una forma distinta. En el caso de un operador español, por ejemplo:
\begin{itemize}
\item {} 
Utilizará 00 + 33 + número para una llamada con origen francés.

\item {} 
Puede que muestre las numeraciones internacionales sin el 00.

\item {} 
Es probable que si la llamada procede del mismo país del operador, no se muestre el prefijo del país (911234567 en lugar de 00 + 34 + 911234567)

\end{itemize}

En el caso de un operador de Ucrania, que no utiliza 00 como código internacional:
\begin{itemize}
\item {} 
Utilizará 810 + 33 + número francés para el mismo número.

\item {} 
Puede incluso que aparte del código internacional (00 en la mayoría de países del mundo) un operador concreto anteceda sus números con un prefijo concreto.

\end{itemize}

El objetivo de la adaptación de entrada es que, independientemente de la notación utilizada por el operador, la llamada se acabe en un formato homogéneo y común.
\phantomsection\label{external_incoming_calls/numeric_transformations:e164}
\begin{notice}{important}{Importante:}
Este formato común se suele llamar E.164 y muestra los números sin el código internacional y con el prefijo de país: e.g. 34911234567
\end{notice}


\subsubsection{Adaptación en salida}
\label{external_incoming_calls/numeric_transformations:adaptacion-en-salida}
Del mismo modo que el origen y el destino se tiene que adaptar en entrada, habrá que adaptar también las numeraciones en salida en función del operador por el que se vaya a sacar la llamada.

Por ejemplo, para una llamada a un número español:
\begin{itemize}
\item {} 
Operador español: El destino vendrá en E164 (34911234567) y habrá que quitar el prefijo de país: 911234567.

\item {} 
Operador francés: El destino vendrá en E164 (34911234567) y habrá que añadir el código internacional: 0034911234567.

\end{itemize}

\begin{notice}{note}{Nota:}
En resumen, consiste en entregar origen y destino en el formato en el que el operador elegido los espera, para que la llamada progrese con normalidad.
\end{notice}

\begin{notice}{tip}{Truco:}
Las transformaciones numéricas utilizan \textbf{expresiones regulares} \footnote[*]{\sphinxAtStartFootnote%
\url{https://es.wikipedia.org/wiki/Expresi\%C3\%B3n\_regular}
} muy simples para describir los cambios a realizar en las numeraciones. En Internet se pueden encontrar múltiples tutoriales sobre el uso básico de expresiones regulares.
\end{notice}


\subsection{Transformación `Operador Nacional'}
\label{external_incoming_calls/numeric_transformations:transformacion-operador-nacional}
IvozProvider incluye un \emph{set} de transformaciones denominado \textbf{Operador nacional} que realiza las modificaciones adecuadas para operadores nacionales que muestren las numeraciones siguiendo estas reglas:
\begin{itemize}
\item {} 
Número español: se muestra sin código internacional y sin el 34.

\item {} 
Número internacional: se precede de 00 y del código del país.

\end{itemize}

Los \emph{sets} de transformaciones numéricas se asignan a \textbf{PeeringContracts}, tal y como veremos en la siguiente sección. Este \emph{set} será el adecuado para la mayoría de operadores españoles.

Analicemos el \emph{set} para entender lo que hace cada regla de transformación:

\noindent\sphinxincludegraphics{{numeric_transformations}.png}


\subsubsection{Adaptación en entrada}
\label{external_incoming_calls/numeric_transformations:id2}
Aparece en azul en la imagen anterior:
\begin{itemize}
\item {} 
Izquierda llamado/destino

\item {} 
Derecha llamante/origen

\end{itemize}

Tanto para el origen como para el destino las reglas que se aplican son las mismas:

\noindent\sphinxincludegraphics{{numeric_transformations2}.png}
\begin{itemize}
\item {} \begin{description}
\item[{Las reglas se evalúan antes o después en función del campo \textbf{métrica} (de menor a mayor).}] \leavevmode\begin{itemize}
\item {} 
Si una regla no \emph{matchea}, se evalúa la siguiente regla.

\item {} 
Si una regla \emph{matchea}, no se evalúan más reglas.

\item {} 
Si ninguna regla \emph{matchea}, no se aplica ningún cambio.

\end{itemize}

\end{description}

\item {} 
El campo \textbf{Buscar} se evalúa contra el campo en cuestión (en el caso de la captura, contra el campo Llamado/Destino de una llamada saliente).
\begin{itemize}
\item {} 
Métrica 1: Que empiece por (\textasciicircum{}) 00 o `+', seguido de un dígito del 1 al 9, seguido de 1 o más dígitos del 0 al 9 hasta el final (\$).

\item {} 
Métrica 2: Que empiece por un dígito del 1 al 9, seguido de 1 o más dígitos del 0 al 9 hasta el final (\$).

\end{itemize}

\item {} 
El campo \textbf{Reemplazar} recoge las capturas del campo buscar (expresadas entre paréntesis, 1 para la primera, 2 para la segunda, etc.) y expresa cómo tiene que quedar el número:
\begin{itemize}
\item {} 
Métrica 1: El campo mostrará solo el segundo elemento capturado (2).

\item {} 
Métrica 1: El campo añadirá 34 al primer elemento capturado (1).

\end{itemize}

\item {} 
Explicación \emph{en modo texto}:
\begin{itemize}
\item {} 
Métrica 1: Quitar código internacional (00 o `+') si lo hay.

\item {} 
Métrica 2: Añadir 34 a los números españoles mostrados sin prefijo de país para que queden en E.164.

\end{itemize}

\end{itemize}


\subsubsection{Adaptación en salida}
\label{external_incoming_calls/numeric_transformations:id3}
\noindent\sphinxincludegraphics{{numeric_transformations3}.png}

Siguiendo la misma lógica, estas 2 reglas realizan los siguientes cambos sobre el destino en llamadas salientes:
\begin{itemize}
\item {} 
Métrica 1: Si empieza con 34 seguido de más dígitos, quitamos 34. Convierte números españoles de E.164 a formato nacional.

\item {} 
Métrica 2: Si no empieza por 34, añadimos 00. Convierte números internacionales de E.164 a formato internacional español.

\end{itemize}

\begin{notice}{attention}{Atención:}
\textbf{En resumen}: las transformaciones numéricas adaptan orígenes y destinos, a E.164 en entrada y a los formatos que los operadores esperan en salida, utilizando reglas con expresiones regulares y métricas agrupadas en \emph{sets} que se asocian a \textbf{PeeringContracts}.
\end{notice}


\section{Configurar Contrato de Peering}
\label{external_incoming_calls/peering_contracts:configurar-contrato-de-peering}\label{external_incoming_calls/peering_contracts::doc}\label{external_incoming_calls/peering_contracts:peering-contracts}
En IvozProvider se entiende por \textbf{Contrato de Peering} el acuerdo entre un \textbf{Operador de Marca} y un Operador VoIP para sacar y recibir llamadas.

IvozProvider permite integrarse con Operadores IP por medio de la sección \textbf{Contratos de Peering} que pasamos a describir:

\noindent{\hspace*{\fill}\sphinxincludegraphics{{peeringcontract_section}.png}\hspace*{\fill}}


\subsection{Datos básicos}
\label{external_incoming_calls/peering_contracts:datos-basicos}
Analicemos los campos de un PeeringContract de ejemplo:

\noindent{\hspace*{\fill}\sphinxincludegraphics{{peeringcontract}.png}\hspace*{\fill}}

Si entramos a editar este Contrato de Peering concreto:

\noindent{\hspace*{\fill}\sphinxincludegraphics{{peeringcontract_edit}.png}\hspace*{\fill}}
\begin{description}
\item[{Nombre\index{Nombre|textbf}}] \leavevmode\phantomsection\label{external_incoming_calls/peering_contracts:term-nombre}
Se utilizará para referencia a este Contrato de Peering.

\item[{Descripción\index{Descripción|textbf}}] \leavevmode\phantomsection\label{external_incoming_calls/peering_contracts:term-descripcion}
Un campo adicional para escribir algún detalle.

\item[{Transformación numérica\index{Transformación numérica|textbf}}] \leavevmode\phantomsection\label{external_incoming_calls/peering_contracts:term-transformacion-numerica}
Transformaciones que se aplicarán al origen y al destino a las numeraciones que entren o salgan por este PeeringContract (ver {\hyperref[external_incoming_calls/numeric_transformations:numeric\string-transformations]{\sphinxcrossref{\DUrole{std,std-ref}{Transformaciones numéricas}}}}).

\item[{Tarificación externa\index{Tarificación externa|textbf}}] \leavevmode\phantomsection\label{external_incoming_calls/peering_contracts:term-tarificacion-externa}
Módulo de tarificación externa disponible solo para ciertos operadores. Permite poner precio a la llamada atacando una API externa en lugar de hacerlo con las lógicas de IvozProvider. Consultar a los {\hyperref[intro/getting_help:getting\string-help]{\sphinxcrossref{\DUrole{std,std-ref}{desarrolladores de la solución}}}} en caso de estar interesados.

\end{description}

\begin{notice}{important}{Importante:}
Los campos marcados con una estrella roja son obligatorios.
\end{notice}


\subsection{PeerServers}
\label{external_incoming_calls/peering_contracts:peerservers}
El concepto de \textbf{PeerServer} se refiere a los distintos servidores SIP que puede tener un Operador IP para servir su dominio SIP. Para definir los PeerServers del OPERADOR que acabamos de definir, hay que pulsar este botón:

\noindent{\hspace*{\fill}\sphinxincludegraphics{{peerservers}.png}\hspace*{\fill}}

Tal y como indica el 0, no hay ningún PeerServer definido, por lo que añadimos uno:

\noindent{\hspace*{\fill}\sphinxincludegraphics{{peerservers_add}.png}\hspace*{\fill}}
\begin{description}
\item[{Nombre\index{Nombre|textbf}}] \leavevmode\phantomsection\label{external_incoming_calls/peering_contracts:term-4}
Se utilizará para referencia a este PeerServer.

\item[{Descripción\index{Descripción|textbf}}] \leavevmode\phantomsection\label{external_incoming_calls/peering_contracts:term-5}
Un campo adicional para escribir algún detalle.

\item[{SIP Proxy\index{SIP Proxy|textbf}}] \leavevmode\phantomsection\label{external_incoming_calls/peering_contracts:term-sip-proxy}
Dirección IP (o registro DNS) del PeerServer. Si utiliza un puerto distinto a 5060, se puede indicar con `:'.

\item[{Esquema URI\index{Esquema URI|textbf}}] \leavevmode\phantomsection\label{external_incoming_calls/peering_contracts:term-esquema-uri}
Los esquemas soportados son sip y sips. Dejar en `sip' en caso de duda.

\item[{Transporte\index{Transporte|textbf}}] \leavevmode\phantomsection\label{external_incoming_calls/peering_contracts:term-transporte}
Los protocolos de transporte SIP soportados. En caso de duda, dejar `udp'.

\item[{Outbound Proxy\index{Outbound Proxy|textbf}}] \leavevmode\phantomsection\label{external_incoming_calls/peering_contracts:term-outbound-proxy}
Normalmente se deja vacío o se pone la IP del dominio indicado en \textbf{SIP Proxy} (para evitar resolución de dominios y hacer que el mensaje SIP contenga el dominio en lugar de una IP). Funciona como un proxy web: en lugar de enviar los mensajes SIP al destino de \textbf{SIP Proxy}, los envía a la IP:PUERTO de este campo.

\item[{Requiere autenticación\index{Requiere autenticación|textbf}}] \leavevmode\phantomsection\label{external_incoming_calls/peering_contracts:term-requiere-autenticacion}
Existen contratos de Peering que nos validarán por IP, otros necesitarán que nos autentiquemos en cada sesión que queramos establecer. En caso de ser del último grupo, este selector nos permite introducir un usuario y una contraseña para responder a esa autenticación.

\item[{Cabecera de origen de llamada\index{Cabecera de origen de llamada|textbf}}] \leavevmode\phantomsection\label{external_incoming_calls/peering_contracts:term-cabecera-de-origen-de-llamada}
Algunos operadores recogen el origen del From. Otros utilizan el From para validar la cuenta de cliente y necesitan cabeceras adicionales para recoger el origen. En caso de duda, marcar \textbf{PAI}.

\item[{Transformaciones R-URI previas a las transformaciones numéricas\index{Transformaciones R-URI previas a las transformaciones numéricas|textbf}}] \leavevmode\phantomsection\label{external_incoming_calls/peering_contracts:term-transformaciones-r-uri-previas-a-las-transformaciones-numericas}
Permiten realizar cambios estáticos al destino de la llamada antes de aplicar las reglas de transformación numéricas mendionadas en {\hyperref[external_incoming_calls/numeric_transformations:numeric\string-transformations]{\sphinxcrossref{\DUrole{std,std-ref}{Transformaciones numéricas}}}}. Se pueden quitar uno dígitos del comienzo, añadir un prefijo después e, incluso, añadir parámetros a la URI siguiendo el formato indicado. En caso de duda, dejar todo el bloque en blanco.

\item[{Personalización de la cabecera From\index{Personalización de la cabecera From|textbf}}] \leavevmode\phantomsection\label{external_incoming_calls/peering_contracts:term-personalizacion-de-la-cabecera-from}
Los operadores que muestren el origen en otras cabeceras (PAI/RPID), es posible que nos soliciten que el From User sea el número de cuenta de cliente y el From Domain (por ejemplo), su dominio SIP. En caso de duda, dejar en blanco.

\end{description}

\begin{notice}{tip}{Truco:}
Existen muchos campos para poder establecer \emph{peering} con operadores de todo tipo, pero lo habitual será poner solo nombre y SIP Proxy (para los operadores que nos validen por IP) o nombre, SIP Proxy y Autenticación.
\end{notice}

\begin{notice}{warning}{Advertencia:}
En caso de definir múltiples PeerServers para un PeeringContract, IvozProvider hará balanceo y failover utilizando todos. Es decir, a veces hablará a uno y otras veces a otro. En caso de que uno no le conteste, lo intentará con el resto hasta dar con uno que conteste.
\end{notice}


\subsection{Registro SIP}
\label{external_incoming_calls/peering_contracts:registro-sip}
Hay operadores que exigen que tengamos un \href{https://tools.ietf.org/html/rfc3261\#section-10}{Registro SIP} activo para que nos metan las llamadas de nuestras numeraciones. Es más, existen operadores que exigen un registro activo para poder sacar llamadas a través de ellos (???).

\begin{notice}{note}{Nota:}
IvozProvider soporta \emph{peerings} de todo tipo, pero recomendamos acordar \emph{peerings de tú a tú}: sin autenticación, sin registro y validados por IP. Esto evita tráfico innecesario (autenticación en cada sesión y registros periódicos) y simplifica la configuración, quedándose casi todo con los valores por defecto.
\end{notice}

Por este motivo IvozProvider permite configurar registros SIP periódicos por medio del siguiente botón:

\noindent\sphinxincludegraphics{{sip_registers}.png}

Si creamos uno nuevo, nos encontramos con la siguiente ventana:

\noindent\sphinxincludegraphics{{sip_registers_add}.png}
\begin{description}
\item[{Nombre de usuario\index{Nombre de usuario|textbf}}] \leavevmode\phantomsection\label{external_incoming_calls/peering_contracts:term-nombre-de-usuario}
Número de cuenta de cliente o similar proporcionada por el operador que exige registro SIP.

\item[{Dominio\index{Dominio|textbf}}] \leavevmode\phantomsection\label{external_incoming_calls/peering_contracts:term-dominio}
Dominio o IP del servidor de registros. Habitualmente el mismo que sirve de Proxy SIP en el PeerServer.

\item[{DDI\index{DDI|textbf}}] \leavevmode\phantomsection\label{external_incoming_calls/peering_contracts:term-ddi}
Se envía en la cabecera Contact y tiene que ser único a nivel de toda la plataforma. En PeeringContracts con un DDI asociado, se recomienda meter ese DDI. En caso de múltiples DDIs asociados, se recomienda meter uno de ellos. En caso de ningún DDI asociado, se recomienda meter un valor único (cualquier valor que te deje guardar).

\item[{Usuario\index{Usuario|textbf}}] \leavevmode\phantomsection\label{external_incoming_calls/peering_contracts:term-usuario}
Usuario de autenticación, practicamente siempre es igual al ``Nombre de usuario'' por lo que se recomienda dejar en blanco.

\item[{URI Servidor de registro\index{URI Servidor de registro|textbf}}] \leavevmode\phantomsection\label{external_incoming_calls/peering_contracts:term-uri-servidor-de-registro}
Normalmente se puede dejar en blanco ya que se deduce del Dominio introducido. Si no fuera así, poner una dirección IP con `sip:' por delante.

\item[{Realm\index{Realm|textbf}}] \leavevmode\phantomsection\label{external_incoming_calls/peering_contracts:term-realm}
Dejar en blanco para aceptar el propuesto por el extremo contrario. Definir solo si de estar familiarizado el mecanismo de autorización de SIP y saber lo que este campo implica.

\item[{Expire\index{Expire|textbf}}] \leavevmode\phantomsection\label{external_incoming_calls/peering_contracts:term-expire}
Tiempo que IvozProvider sugerirá como tiempo de expiración del registro.

\end{description}

\begin{notice}{tip}{Truco:}
Al igual que ocurre con los PeerServers, existen múltiples campos. Hay que tener en cuenta, no obstante, que la mayoría de operadores no deberían de exigir registro y, los que lo hagan, habitualmente solo requerirán de usuario, dominio y contraseña.
\end{notice}

Una vez que hemos llegado a un acuerdo con un Operador VoIP y hemos configurado esta relación de \emph{peering}, solo faltan dos tareas:


\section{Dar de alta un DDI externo}
\label{external_incoming_calls/configure_ddi:settingup-ddi}\label{external_incoming_calls/configure_ddi:dar-de-alta-un-ddi-externo}\label{external_incoming_calls/configure_ddi::doc}
El operador de marca, como único responsable de llegar a acuerdos de \emph{peering} con operadores IP, es el responsable de dar de alta los DDIs de cada operador.

Para ello, tiene que acceder a la siguiente sección:

\noindent{\hspace*{\fill}\sphinxincludegraphics{{ddis_company_section}.png}\hspace*{\fill}}

Observar que para poder acceder a esta sección el operador de marca (o \emph{god}) tiene que haber emulado una empresa concreta y acceder desde el bloque \textbf{Configuración de empresa}.

\begin{notice}{attention}{Atención:}
La sección \textbf{Configuración de empresa \textgreater{} DDIs} es distinta cuando accede un administrador de empresa que cuando accede un operador global o de marca. El administrador de empresa no puede crear nuevos DDIs ni borrar los que ya existen, solo puede editar los que el operador de marca haya creado.
\end{notice}

Por otra parte, la sección \textbf{Configuración de marca \textgreater{} DDIs} solo sirve para que el operador de marca pueda ver los DDIs asociados a sus distintas empresas, es un listado en \emph{read-only}.

\noindent{\hspace*{\fill}\sphinxincludegraphics{{ddis_brand_section}.png}\hspace*{\fill}}

Una vez explicados estos conceptos, añadimos un nuevo DDI y rellenamos los campos necesarios:

\noindent{\hspace*{\fill}\sphinxincludegraphics{{ddis_add}.png}\hspace*{\fill}}
\phantomsection\label{external_incoming_calls/configure_ddi:bill-inbound}\begin{description}
\item[{País\index{País|textbf}}] \leavevmode\phantomsection\label{external_incoming_calls/configure_ddi:term-pais}
El país de la númeración pública que estamos dando de alta.

\item[{DDI\index{DDI|textbf}}] \leavevmode\phantomsection\label{external_incoming_calls/configure_ddi:term-ddi}
El número en cuestión, sin códigos de país.

\item[{Contrato de peering\index{Contrato de peering|textbf}}] \leavevmode\phantomsection\label{external_incoming_calls/configure_ddi:term-contrato-de-peering}
El contrato de peering por el que entra la numeración. Esta relación permite aplicar las {\hyperref[external_incoming_calls/numeric_transformations:numeric\string-transformations]{\sphinxcrossref{\DUrole{std,std-ref}{Transformaciones numéricas}}}} adecuadas.

\item[{Filtro de entrada\index{Filtro de entrada|textbf}}] \leavevmode\phantomsection\label{external_incoming_calls/configure_ddi:term-filtro-de-entrada}
Permite aplicar lógicas de filtrado en base a horario y calendario, se verán en {\hyperref[pbx_features/external_filters:external\string-filters]{\sphinxcrossref{\DUrole{std,std-ref}{Filtros de entrada externo}}}}. Dejar sin seleccionar para no aplicar filtro alguno.

\item[{Enrutar\index{Enrutar|textbf}}] \leavevmode\phantomsection\label{external_incoming_calls/configure_ddi:term-enrutar}
Un DDI puede tener distintos {\hyperref[pbx_features/external_ddis:routing\string-logics]{\sphinxcrossref{\DUrole{std,std-ref}{tratamientos}}}}. Para nuestro objetivo, enrutar al usuario \emph{Alice}.

\item[{Grabar llamadas\index{Grabar llamadas|textbf}}] \leavevmode\phantomsection\label{external_incoming_calls/configure_ddi:term-grabar-llamadas}
Dejar desactivado de momento, se explicará en {\hyperref[pbx_features/call_recording:call\string-recordings]{\sphinxcrossref{\DUrole{std,std-ref}{Grabación de llamadas}}}}.

\item[{Tarificar llamadas entrantes\index{Tarificar llamadas entrantes|textbf}}] \leavevmode\phantomsection\label{external_incoming_calls/configure_ddi:term-tarificar-llamadas-entrantes}
Requiere del módulo de tarificación externa y permite tarificar llamadas entrantes a numeraciones especiales. Consultar a los {\hyperref[intro/getting_help:getting\string-help]{\sphinxcrossref{\DUrole{std,std-ref}{desarrolladores de la solución}}}} en caso de estar interesados.

\end{description}


\section{Configurar tratamiento en entrada}
\label{external_incoming_calls/configure_ddi:configurar-tratamiento-en-entrada}
En la sección anterior hemos dado de alta el DDI y lo hemos configurado, pero conviene tener claro que \textbf{en un uso normal, el administrador de marca} simplemente \textbf{daría de alta el DDI y el administrador de empresa}, accediendo a la misma sección, \textbf{lo configuraría} apuntándolo al usuario / grupo de salto / etc. adecuado, configurando horarios, calendarios, etc.

\begin{notice}{note}{Nota:}
En este punto, marcando el número público debería de sonar el teléfono de \emph{Alice} consiguiendo, por tanto, el objetivo de este bloque :)
\end{notice}


\chapter{Realizar llamadas externas}
\label{external_outgoing_calls/index:realizar-llamadas-externas}\label{external_outgoing_calls/index::doc}
El objetivo de este bloque será configurar IvozProvider para realizar llamadas externas salientes, partiendo de la configuración realizada hasta este momento.

Para ello, seguiremos los siguientes pasos:


\section{¿A dónde llamo?}
\label{external_outgoing_calls/call_types::doc}\label{external_outgoing_calls/call_types:a-donde-llamo}
En este punto de la configuración, tenemos que configurar IvozProvider para que las llamadas a los destinos externos que vayamos a probar salgan por el Contrato de \emph{peering} que hemos configurado en el bloque anterior.

Para ello, en primer lugar, necesitamos que los números externos recaigan en un \textbf{patrón de destino} dado de alta con anterioridad.


\subsection{Patrones de destino}
\label{external_outgoing_calls/call_types:target-patterns}\label{external_outgoing_calls/call_types:patrones-de-destino}
Cuando un usuario marca un número externo, IvozProvider intenta calificar este número en uno de los patrones de destino definidos en esta sección:

\noindent{\hspace*{\fill}\sphinxincludegraphics{{target_patterns_section}.png}\hspace*{\fill}}

Lo más normal será que nos interese tener un patrón de destino por cada uno de los 254 países definidos en la \href{https://es.wikipedia.org/wiki/ISO\_3166}{ISO 3166}. Por ese motivo, IvozProvider incluye estos países y sus prefijos de forma automática:

\noindent{\hspace*{\fill}\sphinxincludegraphics{{target_patterns_default}.png}\hspace*{\fill}}

Dentro de este listado aparece el prefijo de España, que será el grupo del número que probemos en este bloque:

\noindent{\hspace*{\fill}\sphinxincludegraphics{{target_patterns_spain}.png}\hspace*{\fill}}

\begin{notice}{warning}{Advertencia:}
Cada operador de marca puede elegir mantener estos patrones o borrarlos y crear los que le interesen. De hecho, aparte de prefijos, también se pueden definir expresiones regulares. e.g. Queremos crear un único patrón que englobe todas las llamadas: \textasciicircum{}{[}0-9{]}+\$.
\end{notice}

\begin{notice}{danger}{Peligro:}
Crear patrones de destino en base a expresiones regulares puede provocar que un número encaje en 2 patrones. Usar con responsabilidad.
\end{notice}


\subsection{Grupos de patrones}
\label{external_outgoing_calls/call_types:grupos-de-patrones}
Como veremos en la sección de {\hyperref[external_outgoing_calls/call_routing:outgoing\string-routes]{\sphinxcrossref{\DUrole{std,std-ref}{rutas salientes}}}}, cada patrón de destino se vinculará a un Contrato de Peering concreto.

Por este motivo, puede ser interesante agrupar los patrones en grupos y así poder vincular un grupo entero a un Contrato de Peering.

Para ello se utiliza esta sección:

\noindent{\hspace*{\fill}\sphinxincludegraphics{{target_patterngroups_section}.png}\hspace*{\fill}}

Por defecto aparecen los 254 países agrupados en base a su continente definidos en la \href{https://es.wikipedia.org/wiki/ISO\_3166}{ISO 3166}:

\noindent{\hspace*{\fill}\sphinxincludegraphics{{target_patterngroups_default}.png}\hspace*{\fill}}

\begin{notice}{important}{Importante:}
\textbf{En resumen}, cuando un usuario marca un número externo, IvozProvider busca el patrón de destino al que pertenece para saber por dónde tiene que sacar dicha llamada.
\end{notice}

Para conseguir nuestro objetivo de llamar a un número español, no hemos tenido que modificar el contenido de partida de estas dos secciones :)
\phantomsection\label{external_outgoing_calls/call_routing:outgoing-routes}
Ya tenemos nuestra llamada de pruebas categorizada dentro del \textbf{Patrón de destino} `España'. Es más, también tenemos un \textbf{Grupo de patrones de destino} que incluye `España', `Europa'.

Ahora solo nos falta decir a IvozProvider que las llamadas de `España' o de `Europa' salgan por nuestro \textbf{Contrato de peering}.


\section{Rutas salientes}
\label{external_outgoing_calls/call_routing:rutas-salientes}\label{external_outgoing_calls/call_routing::doc}
Para realizar esta vinculación, accedemos a la sección \textbf{Rutas salientes}:

\noindent{\hspace*{\fill}\sphinxincludegraphics{{outgoing_routes_section}.png}\hspace*{\fill}}

Si optamos por enrutar solamente las llamadas de España por nuestro Contrato de \emph{peering}, tendremos que realizar la siguiente configuración:

\noindent{\hspace*{\fill}\sphinxincludegraphics{{outgoing_routes_by_pattern}.png}\hspace*{\fill}}

Por el contrario, si somos más generosos y decidimos permitir todas las llamadas a países europeos, la configuración a aplicar sería la siguiente:

\noindent{\hspace*{\fill}\sphinxincludegraphics{{outgoing_routes_by_patterngroup}.png}\hspace*{\fill}}
\phantomsection\label{external_outgoing_calls/call_routing:routes-metrics}
Existen dos parámetros que merecen explicación:
\begin{description}
\item[{Prioridad\index{Prioridad|textbf}}] \leavevmode\phantomsection\label{external_outgoing_calls/call_routing:term-prioridad}
Si una llamada concreta encaja con rutas de distinta prioridad, la llamada se sacará por la que menor prioridad tenga siempre y cuando esté disponible.

\item[{Métrica\index{Métrica|textbf}}] \leavevmode\phantomsection\label{external_outgoing_calls/call_routing:term-metrica}
Si una llamada concreta encaja con rutas con la misma prioridad, la métrica determina cuántas se sacarán por una ruta y cuántas por otra.

\end{description}

\begin{notice}{note}{Nota:}
Estos dos parámetros son clave para conseguir dos funcionalidades muy interesantes: \textbf{load-balancing} y \textbf{failover-routes}.
\end{notice}


\subsection{Balanceo de carga}
\label{external_outgoing_calls/call_routing:balanceo-de-carga}
El balanceo de carga o \emph{load-balancing} nos permite sacar un porcentaje de llamadas por una ruta y otro porcentaje de llamadas por otra ruta, es decir, nos permite repartir las llamadas entre dos o más rutas igualmente válidas.
\paragraph{Ejemplo 1}
\begin{itemize}
\item {} 
Ruta A: prioridad 1, métrica 1

\item {} 
Ruta B: prioridad 1, métrica 1

\end{itemize}

Las llamadas que encajen en ambas rutas se sacaran el 50\% por una ruta y el 50\% por la otra.
\paragraph{Ejemplo 2}
\begin{itemize}
\item {} 
Ruta A: prioridad 1, métrica 1

\item {} 
Ruta B: prioridad 1, métrica 2

\end{itemize}

Las llamadas que encajen en ambas rutas se sacaran el 33\% por la Ruta A y el 66\% por la Ruta B (el doble por B que por A).


\subsection{Conmutación por error}
\label{external_outgoing_calls/call_routing:conmutacion-por-error}
Las rutas en caso de fallo o \emph{failover-routes} nos permite disponer de ruta adicional en caso de que la ruta preferida falle.
\paragraph{Ejemplo}
\begin{itemize}
\item {} 
Ruta A: prioridad 1, métrica 1

\item {} 
Ruta B: prioridad 2, métrica 1

\end{itemize}

Las llamadas que encajen en ambas rutas se intentarán sacar siempre por la Ruta A. En caso de no error (operador caído, etc.), se intentará sacar por la Ruta B.

\begin{notice}{tip}{Truco:}
Tanto el balanceo de carga como las rutas de fallo permiten encadenar/utilizar más de 2 rutas, aunque en los ejemplo se hayan utilizado solo 2.
\end{notice}


\section{Configurar DDI saliente}
\label{external_outgoing_calls/external_ddi:external-ddi}\label{external_outgoing_calls/external_ddi::doc}\label{external_outgoing_calls/external_ddi:configurar-ddi-saliente}
Antes de realizar la llamada externa, estaría muy bien que dicha llamada se presentara con el DDI que ya hemos configurado en entrada, así el llamado podría devolvernos la llamada cómodamente.

Para ello, basta con configurar dicho DDI como \textbf{DDI saliente} de \emph{Alice}, que será la elegida para realizar la primera llamada saliente de nuestro recién instalado IvozProvider:

\noindent\sphinxincludegraphics{{ddi_out}.png}

Esta configuración se realiza desde \textbf{Configuración de empresa} \textgreater{} \textbf{Usuarios}, editando el usuario de \emph{Alice}. Si ese el operador de marca o el operador global el que realiza esta edición, tendrá que haber {\hyperref[internal_calls/brand_portal:emulate\string-company]{\sphinxcrossref{\DUrole{std,std-ref}{emulado la empresa}}}} previamente.

\begin{notice}{warning}{Advertencia:}
Sin configurar un DDI saliente para el usuario que realiza la llamada, ésta no saldrá al exterior.
\end{notice}

Llegados a este punto y estando deseosos como estamos de hacer nuestra primera llamada, habremos intentando llamar con la configuración actual pero...


\section{Sin plan de precio, no hay llamada}
\label{external_outgoing_calls/noplan_nocall:noplan-nocall}\label{external_outgoing_calls/noplan_nocall:sin-plan-de-precio-no-hay-llamada}\label{external_outgoing_calls/noplan_nocall::doc}
Tal y como advertimos {\hyperref[operation_roles/index:brand\string-responsibilities]{\sphinxcrossref{\DUrole{std,std-ref}{cuando describimos las funciones del operador de marca}}}}, el operador de marca era el \textbf{responsable de realizar la configuración necesaria para que todas las llamadas externas se puedan tarificar}.

\begin{notice}{note}{Nota:}
Se entiende por \textbf{tarificar} la \textbf{acción de poner precio} a una llamada que implique coste.
\end{notice}

Para evitar que por un descuido el operador de marca no defina el precio para un tipo de llamada y llamadas que implican coste salgan a precio 0, \textbf{en el momento del establecimiento de una llamada se comprueba que la llamada se va a poder tarificar}.

\begin{notice}{error}{Error:}
Si una llamada no se va a poder tarificar, IvozProvider no permitirá su establecimiento.
\end{notice}


\subsection{Crear un patrón de precio}
\label{external_outgoing_calls/noplan_nocall:crear-un-patron-de-precio}\label{external_outgoing_calls/noplan_nocall:price-pattern}
Del mismo modo que existen los {\hyperref[external_outgoing_calls/call_types:target\string-patterns]{\sphinxcrossref{\DUrole{std,std-ref}{patrones de destino}}}}, existen los \textbf{patrones de precio} y se configuran en esta sección:

\noindent\sphinxincludegraphics{{pricing_pattern_section}.png}

\begin{notice}{important}{Importante:}
Para que una llamada se considere tarificable, tiene que \emph{matchear} con algún patrón de precio dado de alta previamente.
\end{notice}

A diferencia de los patrones de destino, que vienen precreados con los 254 países del mundo, los patrones de precio no se crean ya que lo habitual es que un país se divida en muchos patrones de precio (redes GSM de un operador, numeraciones especiales, números fijos, números móviles, etc.).

Creemos el patrón de precio \emph{Spain} para nuestra llamada saliente externa:

\noindent\sphinxincludegraphics{{pricing_pattern_add}.png}


\subsection{Crear un plan de precio}
\label{external_outgoing_calls/noplan_nocall:price-plan}\label{external_outgoing_calls/noplan_nocall:crear-un-plan-de-precio}
Un \textbf{Plan de precio} determina el coste de un tipo de llamada (de un patrón de precio) y se configura en esta sección:

\noindent\sphinxincludegraphics{{pricing_plans_section}.png}

Creamos un \textbf{plan de precio} para nuestro objetivo:

\noindent\sphinxincludegraphics{{pricing_plans_add1}.png}

Y le añadimos el \textbf{patrón de precio} que hemos creado antes:

\noindent\sphinxincludegraphics{{pricing_plans_add_price1}.png}

\noindent\sphinxincludegraphics{{pricing_plans_add_price2}.png}

\begin{notice}{note}{Nota:}
Los números decimales se tienen que introducir utilizando el ''.'' como separador decimal (e.g. 0.02)
\end{notice}
\paragraph{Encontrar un plan de precios para un destino concreto}

Para comprobar que hasta este punto la configuración es correcta, podemos \textbf{encontrar un plan de precio} para una llamada concreta pulsando:

\noindent\sphinxincludegraphics{{pricing_plans_find_plan}.png}

Introducimos el número destino en {\hyperref[external_incoming_calls/numeric_transformations:e164]{\sphinxcrossref{\DUrole{std,std-ref}{formato E.164}}}}:

\noindent\sphinxincludegraphics{{pricing_plans_find_plan2}.png}

Y vemos que \emph{matchea} con el \textbf{plan de precio} que acabamos de crear:

\noindent\sphinxincludegraphics{{pricing_plans_find_plan3}.png}


\subsection{Vincular plan de precio a empresa}
\label{external_outgoing_calls/noplan_nocall:pricing-plan-to-company}\label{external_outgoing_calls/noplan_nocall:vincular-plan-de-precio-a-empresa}
Un \textbf{plan de precios} concreto se puede vincular a \emph{n} empresas y corresponde al \emph{Operador de marca} realizar esta vinculación.

Para ello, desde la sección \textbf{Configuración de Marca} \textgreater{} \textbf{Empresas}, seleccionando la empresa \emph{demo}:

\noindent\sphinxincludegraphics{{pricing_plans_relate_company}.png}

La vinculación de \textbf{Planes de precio} y \textbf{Empresas} es efectiva en un período de tiempo concreto, de ahí que haya que seleccionar \emph{Fecha inicio} y \emph{Fecha fin}:

\noindent\sphinxincludegraphics{{pricing_plans_relate_company2}.png}

La \textbf{metrica} de la vinculación permite vincular más de un \emph{plan de precios} a una empresa de forma coincidente en el tiempo, aunque ciertos destinos estén incluidos en más de uno de esos planes de precio.

\begin{notice}{attention}{Atención:}
Una llamada que puede tarificarse en base a más de un plan de precio vinculado a una empresa y activo en un momento dado lo hará en base al plan de precio con menor métrica.
\end{notice}

\begin{notice}{tip}{Truco:}
Esto permite tener un \emph{Plan de precios} general y matizar el coste de algún tipo de llamada en otro \emph{Plan de precio} (móviles gratis, por ejemplo).
\end{notice}
\paragraph{Simular una llamada desde una empresa concreta}

Del mismo modo que al crear un \emph{Plan de precios} hemos comprobado que a nuestra llamada aplicaba dicho plan, podemos simular una llamada concreta en una empresa concreta y saber qué coste se le aplicaría:

\noindent\sphinxincludegraphics{{pricing_plans_simulate_companycall}.png}

Introducimos el número destino en {\hyperref[external_incoming_calls/numeric_transformations:e164]{\sphinxcrossref{\DUrole{std,std-ref}{formato E.164}}}}:

\noindent{\hspace*{\fill}\sphinxincludegraphics{{pricing_plans_simulate_companycall2}.png}\hspace*{\fill}}

Y vemos que efectivamente se le asignaría un precio correcto en base al plan de precio que acabamos de crear y vincular:

\noindent\sphinxincludegraphics{{pricing_plans_simulate_companycall3}.png}

\begin{notice}{note}{Nota:}
En este punto, \emph{Alice} debería de ser capaz de realizar llamadas nacionales externas, que se tendrían que cursar y tarificar con normalidad.
\end{notice}


\chapter{Funcionalidades de PBX}
\label{pbx_features/index::doc}\label{pbx_features/index:funcionalidades-de-pbx}
En el bloque anterior hemos conseguido realizar llamadas internas y externas realizando una configuración mínima, partiendo de la configuración que deja el propio instalador y dejando de lado múltiples apartados y funcionalidades.

El objetivo de este bloque será describir cada una de estas secciones con profundidad y ver así las funcionalidades de PBX que IvozProvider ofrece.

Para ello, seguiremos el siguiente índice:


\section{Extensiones}
\label{pbx_features/extensions:extensiones}\label{pbx_features/extensions:extensions}\label{pbx_features/extensions::doc}
La configuración de partida incluía 2 extensiones (101 y 102) que apuntaban directamente a \emph{Alice} y \emph{Bob}, por lo que apenas dijimos nada sobre la sección \textbf{Configuración de empresa} \textgreater{} \textbf{Extensiones}.

\begin{notice}{note}{Nota:}
\textbf{Una extensión es}, por definición, \textbf{un número interno con una lógica asignada}.
\end{notice}
\paragraph{Crear una nueva extensión}

La ventana de creación de una nueva extensión es tal que así:

\noindent\sphinxincludegraphics{{extension_add}.png}
\begin{description}
\item[{Número\index{Número|textbf}}] \leavevmode\phantomsection\label{pbx_features/extensions:term-numero}
El número que al ser marcado por un usuario interno activará la lógica que sigue. Tiene que tener una longitud mínima de 2 y estar compuesto únicamente por dígitos.

\item[{Enrutar\index{Enrutar|textbf}}] \leavevmode\phantomsection\label{pbx_features/extensions:term-enrutar}
Este selector nos permite indicar la lógica que seguirá esta numeración cuando sea marcada por un usuario concreto. Al seleccionar un ítem en cuestión, se nos mostrará un selector adicional para seleccionar el grupo de salto, sala de conferencias, etc.

\end{description}

\begin{notice}{note}{Nota:}
En el caso de las extensiones que apuntan a usuario, la vinculación con el usuario concreto tendrá que hacerse desde \textbf{Configuración de empresa} \textgreater{} \textbf{Usuarios}.
\end{notice}

\begin{notice}{warning}{Advertencia:}
Si existe una extensión cuyo número conflicte con un número externo, el número externo quedará enmascarado y resultará, en la práctica, inaccesible para toda la empresa.
\end{notice}


\section{Terminales}
\label{pbx_features/terminals:terminals}\label{pbx_features/terminals::doc}\label{pbx_features/terminals:terminales}
La sección \textbf{Configuración de empresa} \textgreater{} \textbf{Terminales} permite dar de alta credenciales SIP que podrán ser utilizados por diversos dispositivos SIP para realizar y recibir llamadas de IvozProvider.

La mejor forma de entender la sección es crear uno nuevo y ver los campos que tenemos que cumplimentar:

\noindent\sphinxincludegraphics{{terminals_add}.png}
\begin{description}
\item[{Nombre\index{Nombre|textbf}}] \leavevmode\phantomsection\label{pbx_features/terminals:term-nombre}
Usuario que utilizará el terminal para presentarse ante IvozProvider y para realizar la fase de autenticación SIP.

\item[{Contraseña\index{Contraseña|textbf}}] \leavevmode\phantomsection\label{pbx_features/terminals:term-contrasena}
Contraseña que utilizará el terminal para responder a la fase de autenticación SIP. Utilizar el generador automático de contraseñas para cumplir los criterios de seguridad exigidos.

\item[{Codecs rechazados/permitidos\index{Codecs rechazados/permitidos|textbf}}] \leavevmode\phantomsection\label{pbx_features/terminals:term-codecs-rechazados-permitidos}
Indica las capacidades del terminal en lo que a codecs se refiere. Primero se descarta los rechazados y posteriormente incluye los soportados. Se recomienda dejar \emph{alaw} como único soportado.

\item[{Modo de actualización\index{Modo de actualización|textbf}}] \leavevmode\phantomsection\label{pbx_features/terminals:term-modo-de-actualizacion}
Indica si el terminal prefiere utilizar reINVITEs o UPDATEs para actualizar la sesión. La sección de ayuda indica qué terminales suelen requerir qué método. En caso de duda, utilizar \emph{invite}.

\item[{Modelo de terminal\index{Modelo de terminal|textbf}}] \leavevmode\phantomsection\label{pbx_features/terminals:term-modelo-de-terminal}
Indica el tipo de provisión que tendrá que recibir este terminal concreto. En el apartado dedicado a la {\hyperref[provisioning/index:provisioning]{\sphinxcrossref{\DUrole{std,std-ref}{provisión de terminales}}}} se verá que existen unos modelos con provisión automática y se explicará todo en profundidad. En caso de no necesitar provisión, utilizar \emph{Generic}.

\item[{MAC\index{MAC|textbf}}] \leavevmode\phantomsection\label{pbx_features/terminals:term-mac}
Campo necesario para modelos que utilizan el sistema de {\hyperref[provisioning/index:provisioning]{\sphinxcrossref{\DUrole{std,std-ref}{provisión de terminales}}}} de IvozProvider. Recoge la \href{https://es.wikipedia.org/wiki/Direcci\%C3\%B3n\_MAC}{dirección física} del adaptador de red del dispositivo SIP.

\end{description}

\begin{notice}{note}{Nota:}
\textbf{En la mayoría} de dispositivos, siempre que no requieran provisión, \textbf{bastará con rellenar el nombre y contraseña}.
\end{notice}

\begin{notice}{hint}{Consejo:}
Una vez dado de alta el Terminal, en la mayoría de dispositivos bastará con configurar nombre, contraseña y {\hyperref[internal_calls/brand_portal:domain\string-per\string-company]{\sphinxcrossref{\DUrole{std,std-ref}{dominio SIP de la compañía}}}} para poder llamar.
\end{notice}


\section{Permisos de llamada}
\label{pbx_features/call_permissions:permisos-de-llamada}\label{pbx_features/call_permissions:call-permissions}\label{pbx_features/call_permissions::doc}
Los \textbf{Permisos de llamada} determinan qué usuarios pueden llamar a qué destinos internos.

\begin{notice}{attention}{Atención:}
Las extensiones internas son accesibles por todos los usuarios, \textbf{los permisos de llamada solo aplican a llamadas externas}.
\end{notice}

La configuración de \textbf{permisos de llamada} tiene 2 partes:
\begin{itemize}
\item {} 
Clasificar las llamadas en determinados tipos utilizando \textbf{expresiones regulares} \footnote[*]{\sphinxAtStartFootnote%
\url{https://es.wikipedia.org/wiki/Expresi\%C3\%B3n\_regular}
}:
\begin{itemize}
\item {} 
A nivel de marca: \textbf{Configuración de Marca} \textgreater{} \textbf{Patrones de permiso genéricos}.

\item {} 
A nivel de empresa: \textbf{Configuración de Empresa} \textgreater{} \textbf{Patrones de permisos de llamada}.

\end{itemize}

\item {} 
Definir políticas para los patrones deseados: \textbf{Configuración de Empresa} \textgreater{} \textbf{Permisos llamada}

\end{itemize}


\subsection{Patrones de permiso}
\label{pbx_features/call_permissions:patrones-de-permiso}
La clasificación de llamadas se realiza en las secciones indicadas a nivel de marca o a nivel de empresa.

\begin{notice}{note}{Nota:}
Cuando un operador de marca crea una empresa, todos los \textbf{Patrones de permiso genéricos} definidos en ese momento a nivel de \textbf{Configuración de Marca} se copian a \textbf{Configuración de Empresa} \textgreater{} \textbf{Patrones de permisos de llamada}. De esta forma, el operador de marca puede definir los más habituales y evitar este trabajo a los administradores de empresa.
\end{notice}

La creación de nuevos patrones es muy simple:

\noindent\sphinxincludegraphics{{permissions_patterns}.png}

Este nuevo patrón de permisos engloba las llamadas que comiencen por 6 o 7 y sigan con 8 dígitos del 0 al 9. Es decir, engloba todas las numeraciones móviles nacionales.

Otros patrones de permisos que pueden resultar interesantes son:
\begin{itemize}
\item {} \begin{description}
\item[{Fijos nacionales (incluyendo numeraciones especiales: 902, etc.): \textasciicircum{}{[}89{]}{[}0-9{]}\{8\}\$}] \leavevmode\begin{itemize}
\item {} 
Un 8 o un 9 seguido de 8 dígitos.

\end{itemize}

\end{description}

\item {} \begin{description}
\item[{Fijos nacionales (excluyendo numeraciones especiales: 902, etc.): \textasciicircum{}{[}89{]}{[}1-9{]}{[}0-9{]}\{7\}\$}] \leavevmode\begin{itemize}
\item {} 
Un 8 o un 9, seguido de 1 dígito del 1 al 9, seguido de 7 dígitos.

\end{itemize}

\end{description}

\item {} \begin{description}
\item[{Llamadas internacionales: \textasciicircum{}00{[}0-9{]}+\$}] \leavevmode\begin{itemize}
\item {} 
00 (código internacional) seguido de 1 o más dígitos.

\end{itemize}

\end{description}

\item {} \begin{description}
\item[{Llamadas a Reino Unido: \textasciicircum{}0044{[}0-9{]}+\$}] \leavevmode\begin{itemize}
\item {} 
00 (código internacional), 44 (código de UK), seguido de 1 o más dígitos.

\end{itemize}

\end{description}

\end{itemize}
\paragraph{Formato de los números externos}

Las expresiones regulares de los patrones de permiso tienen que utilizar el formato que utilice un usuario del mismo país que la compañía a la hora de llamar.

Es decir, un usuario (español, porque tal y como veremos los usuarios tienen también un código de país) de una empresa de española llamará a números móviles, por ejemplo, así: 676 676 676.

\begin{notice}{danger}{Peligro:}
No se utiliza código de país ni formato E.164, \textbf{se escriben las expresiones regulares tal y como llamaría un usuario con el mismo código de país que la empresa en cuestión}.
\end{notice}


\subsection{Permisos de llamada}
\label{pbx_features/call_permissions:id2}
La definición de un \textbf{Permiso de llamada} utiliza una lógica más fácil de describir con un ejemplo.

Imaginemos que tenemos los siguientes \textbf{patrones de permisos}:

\noindent\sphinxincludegraphics{{permissions_patterns_list}.png}

Podríamos definir un \textbf{Permiso de llamada} como el que sigue:

\noindent\sphinxincludegraphics{{permissions_add}.png}

\begin{notice}{note}{Nota:}
La acción por defecto describe lo que hay que hacer con la llamada una vez que se han evaluado todas las reglas (permitir/rechazar).
\end{notice}

Editamos el \textbf{permiso de llamada} que acabamos de crear para añadir las reglas necesarias:

\noindent\sphinxincludegraphics{{permissions_add2}.png}

La \textbf{métrica} determina el orden de evaluación de las reglas y la acción lo que se hará con la llamada en caso de \emph{matchear} (permitir/rechazar).

\noindent\sphinxincludegraphics{{permissions_add3}.png}

Una vez hecho lo propio para el otro \textbf{patrón de permisos}, nuestro \textbf{permiso de llamadas} quedará como sigue:

\noindent\sphinxincludegraphics{{permissions_add4}.png}

Ya solo faltaría añadírselo a un usuario concreto en la sección \textbf{Configuración de empresa} \textgreater{} \textbf{Usuarios}:

\noindent\sphinxincludegraphics{{permissions_add5}.png}

Desde este momento, Alice solo podría llamar a extensiones internas (siempre están permitidas) y a móviles y fijos nacionales.


\section{Usuarios}
\label{pbx_features/users:usuarios}\label{pbx_features/users::doc}\label{pbx_features/users:users}
El proceso de instalación nos creó a \emph{Alice} y a \emph{Bob}, y esto nos permitió ahorrar mucho tiempo a la hora de conseguir que se llamarán entre ellos.

También nos permitió pasar un poco de puntillas por la sección de \textbf{Usuarios}, que ahora pasamos a definir en profundidad.

Para ello, nada mejor que editar el usuario \emph{Alice} y describir cada campo del mismo, bloque a bloque:


\subsection{Datos personales}
\label{pbx_features/users:datos-personales}
\noindent\sphinxincludegraphics{{users_edit_personal_data}.png}
\begin{description}
\item[{Nombre\index{Nombre|textbf}}] \leavevmode\phantomsection\label{pbx_features/users:term-nombre}
Se utilizará para referenciar al usuario en múltiples ámbitos, incluyendo el nombre que se presenta en llamadas internas.

\item[{Apellidos\index{Apellidos|textbf}}] \leavevmode\phantomsection\label{pbx_features/users:term-apellidos}
Irá junto con el Nombre en casi todos los casos.

\item[{Email\index{Email|textbf}}] \leavevmode\phantomsection\label{pbx_features/users:term-email}
Dirección de correo electrónico del usuario al que se enviarán los mensajes de voz del buzón.

\end{description}


\subsection{Credenciales portal de usuario}
\label{pbx_features/users:credenciales-portal-de-usuario}
\noindent\sphinxincludegraphics{{users_edit_login}.png}
\begin{description}
\item[{Nombre de usuario\index{Nombre de usuario|textbf}}] \leavevmode\phantomsection\label{pbx_features/users:term-nombre-de-usuario}
Nombre de usuario para acceder al {\hyperref[userportal/index:userportal]{\sphinxcrossref{\DUrole{std,std-ref}{portal de usuario}}}}.

\item[{Contraseña\index{Contraseña|textbf}}] \leavevmode\phantomsection\label{pbx_features/users:term-contrasena}
Contraseña para acceder al {\hyperref[userportal/index:userportal]{\sphinxcrossref{\DUrole{std,std-ref}{portal de usuario}}}}.

\item[{Activo\index{Activo|textbf}}] \leavevmode\phantomsection\label{pbx_features/users:term-activo}
Da posibilidad al administrador de la plataforma de desactivar el acceso al {\hyperref[userportal/index:userportal]{\sphinxcrossref{\DUrole{std,std-ref}{portal de usuario}}}}.

\end{description}


\subsection{Configuración básica}
\label{pbx_features/users:configuracion-basica}
\noindent\sphinxincludegraphics{{users_edit_basic_config}.png}
\begin{description}
\item[{Terminal\index{Terminal|textbf}}] \leavevmode\phantomsection\label{pbx_features/users:term-terminal}
Los terminales dados de alta en {\hyperref[pbx_features/terminals:terminals]{\sphinxcrossref{\DUrole{std,std-ref}{Terminales}}}} se listan aquí para realizar la vinculación usuario-terminal.

\item[{Extensión\index{Extensión|textbf}}] \leavevmode\phantomsection\label{pbx_features/users:term-extension}
Tal y como se explicaba en {\hyperref[pbx_features/extensions:extensions]{\sphinxcrossref{\DUrole{std,std-ref}{Extensiones}}}}, fija la extensión del usuario para llamadas internas.

\item[{DDI de salida\index{DDI de salida|textbf}}] \leavevmode\phantomsection\label{pbx_features/users:term-ddi-de-salida}
Adelantado en {\hyperref[external_outgoing_calls/external_ddi:external\string-ddi]{\sphinxcrossref{\DUrole{std,std-ref}{Configurar DDI saliente}}}}, determina el número que presenta el usuario en llamadas externas salientes.

\item[{Permiso de llamada\index{Permiso de llamada|textbf}}] \leavevmode\phantomsection\label{pbx_features/users:term-permiso-de-llamada}
Asignación de un grupo de permisos explicado en profundidad {\hyperref[pbx_features/call_permissions:call\string-permissions]{\sphinxcrossref{\DUrole{std,std-ref}{aquí}}}}.

\item[{No molestar\index{No molestar|textbf}}] \leavevmode\phantomsection\label{pbx_features/users:term-no-molestar}
Impide que nadie pueda llamar a este usuario, sin impedirle a él llamar a donde desee.

\item[{Llamada en espera\index{Llamada en espera|textbf}}] \leavevmode\phantomsection\label{pbx_features/users:term-llamada-en-espera}
Determina si el sistema tendrá que llamar al usuario cuando éste ya esté en conversación.

\end{description}


\subsection{Buzón de voz}
\label{pbx_features/users:buzon-de-voz}
\noindent\sphinxincludegraphics{{users_edit_vm}.png}
\begin{description}
\item[{Buzón de voz\index{Buzón de voz|textbf}}] \leavevmode\phantomsection\label{pbx_features/users:term-buzon-de-voz}
Activa o desactiva \textbf{la existencia} del buzón de voz del usuario. Aparte de existir, como veremos {\hyperref[pbx_features/users:fwd\string-to\string-vm]{\sphinxcrossref{\DUrole{std,std-ref}{más adelante}}}}, habrá que desviar las llamadas que queramos al buzón.

\item[{Notificar por email\index{Notificar por email|textbf}}] \leavevmode\phantomsection\label{pbx_features/users:term-notificar-por-email}
Mandar a la dirección del usuario un correo notificando del mensaje de voz.

\item[{Adjuntar audio del mensaje\index{Adjuntar audio del mensaje|textbf}}] \leavevmode\phantomsection\label{pbx_features/users:term-adjuntar-audio-del-mensaje}
Adjuntar o no en dicho mail el audio del mensaje

\end{description}


\subsection{Jefe-Asistente}
\label{pbx_features/users:jefe-asistente}
\noindent\sphinxincludegraphics{{users_edit_boss}.png}

La funcionalidad jefe-asistente impide que un usuario sea molestado por nadie que no sea:
\begin{itemize}
\item {} 
Su asistente.

\item {} 
Las excepciones que él defina.

\end{itemize}

Toda llamada con destino a \emph{un jefe} será desviada al asistente.
\begin{description}
\item[{Jefe\index{Jefe|textbf}}] \leavevmode\phantomsection\label{pbx_features/users:term-jefe}
Indica que el usuario en cuestión es jefe o no.

\item[{Asistente\index{Asistente|textbf}}] \leavevmode\phantomsection\label{pbx_features/users:term-asistente}
Si el usuario es jefe, indica quién es su asistente.

\item[{Whitelist jefe/asistente\index{Whitelist jefe/asistente|textbf}}] \leavevmode\phantomsection\label{pbx_features/users:term-whitelist-jefe-asistente}
Permite indicar utilizando expresiones regulares las excepciones deseadas.

\end{description}

Con la configuración de la imagen, toda llamada a \emph{Alice} acabará en \emph{Bob}, salvo las que haga el propio \emph{Bob} y el número de teléfono 945 945 945.


\subsection{Pertenencia a grupos}
\label{pbx_features/users:pertenencia-a-grupos}
\noindent\sphinxincludegraphics{{users_edit_groups}.png}

Tal y como se verá en las secciones {\hyperref[pbx_features/huntgroups:huntgroups]{\sphinxcrossref{\DUrole{std,std-ref}{Grupos de salto}}}} y {\hyperref[pbx_features/call_captures:capture\string-groups]{\sphinxcrossref{\DUrole{std,std-ref}{Capturas de llamada}}}}, un usuario puede pertenecer a uno o a varios grupos de captura y grupos de salto.

Aparte de poder configurar dicha pertenencia desde las propias secciones {\hyperref[pbx_features/huntgroups:huntgroups]{\sphinxcrossref{\DUrole{std,std-ref}{Grupos de salto}}}} y {\hyperref[pbx_features/call_captures:capture\string-groups]{\sphinxcrossref{\DUrole{std,std-ref}{Capturas de llamada}}}}, se puede añadir al usuario que estamos editando a los grupos de captura y grupos de salto deseados que ya existan.

En el caso de la pertenencia a \textbf{grupos de salto}, también se puede configurar desde el listado general:

\noindent{\hspace*{\fill}\sphinxincludegraphics{{users_huntgroups}.png}\hspace*{\fill}}


\subsection{Desvíos de llamada}
\label{pbx_features/users:desvios-de-llamada}
Los desvíos de las llamadas de un usuario concreto se configura pulsando el siguiente botón:

\noindent{\hspace*{\fill}\sphinxincludegraphics{{users_call_fwd}.png}\hspace*{\fill}}
\phantomsection\label{pbx_features/users:fwd-to-vm}
Para desviar las llamadas externas que no se contesten en 15 segundos al buzón de voz que acabamos de configurar, por ejemplo, habría que configurar un desvío tal que:

\noindent{\hspace*{\fill}\sphinxincludegraphics{{users_call_fwd2}.png}\hspace*{\fill}}

Estos son los campos y los posibles valores:
\begin{description}
\item[{Tipo de llamada\index{Tipo de llamada|textbf}}] \leavevmode\phantomsection\label{pbx_features/users:term-tipo-de-llamada}
Limita el desvío a cierto tipo de llamadas, a elegir entre externa, interna o ambas

\item[{Tipo de desvío\index{Tipo de desvío|textbf}}] \leavevmode\phantomsection\label{pbx_features/users:term-tipo-de-desvio}\begin{description}
\item[{Indica cuándo aplica el desvío:}] \leavevmode\begin{itemize}
\item {} 
Incondicional: siempre

\item {} 
Perdida: cuando no se conteste al de X segundos

\item {} 
Ocupado: cuando el usuario esté ocupado (hablando o con el \emph{No molestar} activo)

\item {} 
No registrado: cuando el usuario tenga su terminal sin conectar con IvozProvider

\end{itemize}

\end{description}

\item[{Tipo de destino\index{Tipo de destino|textbf}}] \leavevmode\phantomsection\label{pbx_features/users:term-tipo-de-destino}\begin{description}
\item[{Indica a dónde se enviará la llamada cuando el desvío aplique:}] \leavevmode\begin{itemize}
\item {} 
Buzón de voz

\item {} 
Número (externo)

\item {} 
Extensión (interna)

\end{itemize}

\end{description}

\end{description}

\begin{notice}{hint}{Consejo:}
Si queremos que desviar a un grupo de salto, por ejemplo, bastaría con crear una extensión que apunte al grupo de salto deseado y seleccionar \emph{Extensión} en el \textbf{Tipo de destino}.
\end{notice}


\section{Música en espera}
\label{pbx_features/music_on_hold::doc}\label{pbx_features/music_on_hold:musica-en-espera}\label{pbx_features/music_on_hold:musiconhold}
La música en espera suena cuando un usuario retiene la llamada y su interlocutor queda a la espera de seguir la conversación.

IvozProvider permite definir la música que sonará en estos momentos en 2 niveles distintos:
\begin{itemize}
\item {} 
A nivel de operador de marca: \textbf{Configuración de Marca} \textgreater{} \textbf{Músicas en espera genéricas}

\item {} 
A nivel de administrador de empresa: \textbf{Configuración de Empresa} \textgreater{} \textbf{Música en espera}

\end{itemize}

Si una empresa define su música en espera, sonará. Si, por el contrario, no define ninguna, sonará la que haya definido el operador de su marca.

\begin{notice}{note}{Nota:}
Se pueden definir más de una música en espera y el sistema elegirá de forma aleatoria la música en espera para cada llamada.
\end{notice}
\paragraph{Añadir una nueva música en espera}

\noindent\sphinxincludegraphics{{moh_add}.png}

Una vez \emph{encodeada} el campo \textbf{Status} muestra que está \emph{ready} y se utilizará en las próximas llamadas:

\noindent\sphinxincludegraphics{{moh_add2}.png}

\begin{notice}{tip}{Truco:}
IvozProvider reconoce archivos de audio en los formatos más comunes y los \emph{encodea} a los formatos óptimos para la plataforma.
\end{notice}

Después de \emph{encodear}, podemos descargar el archivo original y el convertido sin más que editar el ítem:

\noindent\sphinxincludegraphics{{moh_add3}.png}


\section{Locuciones}
\label{pbx_features/sounds::doc}\label{pbx_features/sounds:locuciones}
Las locuciones de la plataforma se crean y se suben de forma idéntica a cómo lo hacen las {\hyperref[pbx_features/music_on_hold:musiconhold]{\sphinxcrossref{\DUrole{std,std-ref}{músicas en espera}}}}.

La sección que permite al administrador de empresa subir las locuciones que utilizará en diversos puntos de la configuración (IVR, etc.) es \textbf{Configuración de Empresa} \textgreater{} \textbf{Locuciones}.

\noindent\sphinxincludegraphics{{sounds}.png}

\begin{notice}{hint}{Consejo:}
La única diferencia entre una \textbf{locución} y una \textbf{música en espera} es su uso: la primera sonará cuando el administrador de empresa lo indique (fuera de horario, IVRs, etc.) y la segunda, en cambio, cuando la llamada sea retenida por un usuario.
\end{notice}


\section{Filtros de entrada externo}
\label{pbx_features/external_filters:filtros-de-entrada-externo}\label{pbx_features/external_filters:external-filters}\label{pbx_features/external_filters::doc}
Una de las configuraciones más habituales que todo administrador de empresa querrá realizar en un momento dado es poner filtros de horario y de calendario a sus {\hyperref[pbx_features/external_ddis:external\string-ddis]{\sphinxcrossref{\DUrole{std,std-ref}{DDIs externos}}}}.

Para ello, el primer paso es crear un horario.
\paragraph{Crear un horario}

La sección \textbf{Configuración de Empresa} \textgreater{} \textbf{Horarios} permite definir los tramos horarios en los que un {\hyperref[pbx_features/external_ddis:external\string-ddis]{\sphinxcrossref{\DUrole{std,std-ref}{DDI externo}}}} está dentro de horario.

La interfaz que se le presenta al administrador de empresa cuando añade un horario es la siguiente:

\noindent\sphinxincludegraphics{{timetables_add}.png}

Con la configuración anterior, hemos definido el tramo horario de mañana de una empresa de Lunes a Jueves.

Sigamos con el ejemplo y definamos el tramo horario de tarde de Lunes a Jueves:

\noindent\sphinxincludegraphics{{timetables_add2}.png}

Y el horario reducido de los viernes:

\noindent\sphinxincludegraphics{{timetables_add3}.png}

Ya tenemos los siguientes tramos horarios que, combinados, describen un horario de oficina tradicional:

\noindent\sphinxincludegraphics{{timetables_add4}.png}

\begin{notice}{warning}{Advertencia:}
Los horarios definen, una vez sumados, los tramos horarios activos: toda hora que no caiga dentro de uno de los tramos agrupados será considerada como llamada fuera de horario.
\end{notice}
\paragraph{Crear un calendario}

Los calendarios definen, una vez sumados, los días que se tienen que considerar festivos.

Imaginemos que creamos 3 calendarios tal que:

\noindent\sphinxincludegraphics{{calendars_list}.png}

La creación de calendarios solo requiere dar un nombre al mismo, una vez creado hay que añadir días festivos a dicho calendario pulsando el botón de la derecha del lápiz:

\noindent\sphinxincludegraphics{{calendars_add_day}.png}

Desde este momento, es calendario tiene marcado el día 1 de Enero de 2016 como día festivo con locución de festivo especial llamada ``Feliz año''.

\begin{notice}{warning}{Advertencia:}
Los calendarios tienen una lógica contraria a los horarios: si un día concreto no está definido como festivo en uno de los calendarios, será considerado como día laborable.
\end{notice}

\begin{notice}{hint}{Consejo:}
Si un día es festivo en un calendario y no hay locución de festivo especial, se reproducirá la locución de festivo del filtro de entrada externo (ver más abajo).
\end{notice}
\paragraph{Crear un filtro de entrada externo}

Una vez que tenemos unos horaris y unos calendarios creados, llega el momento de agruparlos en lo que en IvozProvider se conoce por \textbf{Filtro de entrada externos}.

El interfaz que se le presenta al administrador de empresa es el siguiente:

\noindent\sphinxincludegraphics{{external_call_filter}.png}
\begin{description}
\item[{Nombre\index{Nombre|textbf}}] \leavevmode\phantomsection\label{pbx_features/external_filters:term-nombre}
Nombre con el que se referenciará este filtro.

\item[{Locución de bienvenida\index{Locución de bienvenida|textbf}}] \leavevmode\phantomsection\label{pbx_features/external_filters:term-locucion-de-bienvenida}
Esta locución se reproduce siempre que la llamada no se va a rechazar por festivo o fuera horario (es decir, en una hora normal de un día normal).

\item[{Locución de festivo\index{Locución de festivo|textbf}}] \leavevmode\phantomsection\label{pbx_features/external_filters:term-locucion-de-festivo}
Esta locución se reproducirá cuando el día de hoy esté marcado como festivo en uno de los calendarios y no se haya definido una locución de festivo especial en dicho calendario para ese día.

\item[{Tipo de desvío festivo\index{Tipo de desvío festivo|textbf}}] \leavevmode\phantomsection\label{pbx_features/external_filters:term-tipo-de-desvio-festivo}
Si una llamada entra un día festivo, aparte de la locución de festivo (si la hay), se puede realizar un desvío a un buzón de voz, a un número externo o a una extensión interna. En el ejemplo, la llamada se desviará los días festivos al móvil 676 676 676.

\item[{Locución fuera horario\index{Locución fuera horario|textbf}}] \leavevmode\phantomsection\label{pbx_features/external_filters:term-locucion-fuera-horario}
Esta locución se reproducirá cuando, siendo un día no festivo, la hora actual no esté dentro de ningunos de los horarios vinculados.

\item[{Tipo de desvío fuera horario\index{Tipo de desvío fuera horario|textbf}}] \leavevmode\phantomsection\label{pbx_features/external_filters:term-tipo-de-desvio-fuera-horario}
Lo mismo que en festivo, pero para el caso de fuera horario. En la imagen, no se ejecutaría ningún desvío.

\item[{Calendarios\index{Calendarios|textbf}}] \leavevmode\phantomsection\label{pbx_features/external_filters:term-calendarios}
Permite seleccionar múltiples calendarios, de forma que el calendario del filtro es la suma de todos ellos.

\item[{Horarios\index{Horarios|textbf}}] \leavevmode\phantomsection\label{pbx_features/external_filters:term-horarios}
Permite seleccionar múltiples horarios, de forma que el horario del filtro es la suma de todos ellos.

\end{description}

\begin{notice}{attention}{Atención:}
El día festivo tiene prevalencia sobre el fuera horario. Primero se evalúan los calendarios, luego los horarios.
\end{notice}

En la siguiente sección veremos como este filtro se asigna a los {\hyperref[pbx_features/external_ddis:external\string-ddis]{\sphinxcrossref{\DUrole{std,std-ref}{DDIs externos}}}} que queramos, definiendo así el comportamiento de los mismos fuera de horario y en días festivos, así como dentro de horario en el caso de \emph{setear} locución de bienvenida en el filtro.


\section{Grupos de salto}
\label{pbx_features/huntgroups:grupos-de-salto}\label{pbx_features/huntgroups::doc}\label{pbx_features/huntgroups:huntgroups}
Los grupos de salto nos permiten definir lógicas de \emph{ringado} más allá de la básica \textbf{llamada a usuario}.

Existen de varios tipos:
\begin{description}
\item[{A todos\index{A todos|textbf}}] \leavevmode\phantomsection\label{pbx_features/huntgroups:term-a-todos}
La llamada suena en varios terminales a la vez durante el tiempo definido.

\item[{Secuencialmente (una vez)\index{Secuencialmente (una vez)|textbf}}] \leavevmode\phantomsection\label{pbx_features/huntgroups:term-secuencialmente-una-vez}
Suenan los usuarios definidos, en el orden definido y durante el tiempo definido a cada uno de ellos. Al acabar la secuencia, la llamada se cuelga.

\item[{Secuencialmente (infito)\index{Secuencialmente (infito)|textbf}}] \leavevmode\phantomsection\label{pbx_features/huntgroups:term-secuencialmente-infito}
Suenan los usuarios definidos, en el orden definido y durante el tiempo definido. Al acabar la secuencia, se vuelve a iniciar la secuencia.

\item[{Aleatoriamente\index{Aleatoriamente|textbf}}] \leavevmode\phantomsection\label{pbx_features/huntgroups:term-aleatoriamente}
Reparte las llamadas entre los usuarios elegidos de forma aleatoria, sonando cada uno de ellos el tiempo acordado. Una vez que suenan todos, la llamada se cuelga.

\end{description}
\paragraph{Ejemplo 1: Grupo de salto \emph{ringall}}

Creemos por ejemplo un grupo de salto que llame a la par a nuestros 2 usuarios durante 30 segundos:

\noindent\sphinxincludegraphics{{huntgroup_add}.png}

Pulsando el icono de las personas podemos añadir a Alice y Bob a nuestro grupo de salto:

\noindent\sphinxincludegraphics{{huntgroup_add2}.png}
\paragraph{Ejemplo 2: Grupo de salto secuencial}

Editemos ahora este grupo de salto para que llame 10 segundos a Alice y después 15 segundos a Bob, reiniciando la secuencia si ninguno de los 2 contesta:

\noindent\sphinxincludegraphics{{huntgroup_add3}.png}

En este caso tenemos que indicar una \textbf{prioridad} (los usuarios suenan de menos a mayor prioridad) y un tiempo de \emph{ringing}:

\noindent\sphinxincludegraphics{{huntgroup_add4}.png}

\begin{notice}{hint}{Consejo:}
Para que ciertas lógicas (desvíos, etc.) alcancen un \textbf{Grupo de salto}, basta con crear una extensión apuntando a dicho grupo de salto y utilizar esta extensión como destino de la lógica.
\end{notice}

Creemos la extensión 200 que apunte a este grupo de salto:

\noindent\sphinxincludegraphics{{huntgroup_extension}.png}


\section{Interactive Voice Response (IVR)}
\label{pbx_features/ivrs::doc}\label{pbx_features/ivrs:interactive-voice-response-ivr}
Un IVR es la forma habitual de realizar \textbf{menús telefónicos} en los que \textbf{el llamante puede decidir el destino de su llamada sin más que pulsar dígitos} en base a las locuciones que se le van reproduciendo.


\subsection{IVRs genéricos}
\label{pbx_features/ivrs:generic-ivrs}\label{pbx_features/ivrs:ivrs-genericos}
En los IVR genéricos el llamante marca directamente una extensión interna que conoce (o que se le indica en la locución de bienvenida) y el sistema le pone en contacto con dicha extensión de forma automática:

\noindent\sphinxincludegraphics{{ivr_generic}.png}

Describamos cada uno de los parámetros de un IVR genérico:
\begin{description}
\item[{Nombre\index{Nombre|textbf}}] \leavevmode\phantomsection\label{pbx_features/ivrs:term-nombre}
Forma de referenciar el IVR

\item[{Timeout\index{Timeout|textbf}}] \leavevmode\phantomsection\label{pbx_features/ivrs:term-timeout}
Tiempo adicional al de la locución de bienvenida que el sistema espera a que el interlocutor marque los dígitos.

\item[{Locución de bienvenida\index{Locución de bienvenida|textbf}}] \leavevmode\phantomsection\label{pbx_features/ivrs:term-locucion-de-bienvenida}
Locución que se le presenta al llamante invitándole a marcar la extensión con la que quiera hablar.

\item[{Locución de éxito\index{Locución de éxito|textbf}}] \leavevmode\phantomsection\label{pbx_features/ivrs:term-locucion-de-exito}
En caso de que la extensión marcada sea correcta, se reproducirá esta locución (típicamente dirá algo así como ``Contantando...'').

\item[{Expresión regular lista negra\index{Expresión regular lista negra|textbf}}] \leavevmode\phantomsection\label{pbx_features/ivrs:term-expresion-regular-lista-negra}
Si queremos que alguna extensión interna no sea accesible desde el IVR, basta con añadirla aquí utilizando expresiones regulares. En el ejemplo, las extensiones 105 y 106 no serían accesibles desde el IVR y marcarlas provocaría que se activase la \textbf{Configuración de Error}.

\item[{Configuración No contesta\index{Configuración No contesta|textbf}}] \leavevmode\phantomsection\label{pbx_features/ivrs:term-configuracion-no-contesta}
En caso de que la extensión destino no conteste en el tiempo indicado, se reproduce la locución indicada y se desvía al número externo, extensión interna o buzón seleccionado.

\item[{Configuración Error\index{Configuración Error|textbf}}] \leavevmode\phantomsection\label{pbx_features/ivrs:term-configuracion-error}
Si lo que ha marcado el llamante no es válido (o no ha marcado nada), se reproduce la locución indicada y se desvía la llamada al número externo, extensión interna o buzón seleccionado.

\end{description}


\subsection{IVRs a medida}
\label{pbx_features/ivrs:ivrs-a-medida}\label{pbx_features/ivrs:custom-ivrs}
A diferencia de los IVRs genéricos en donde el llamante solo puede marcar extensiones internas, los IVRs a medida permiten marcar dígitos que luego se pueden convertir a lo que el administrador de empresa desee.

\begin{notice}{hint}{Consejo:}
El caso más típico es el IVR que dice algo así como ``Marque 1 si quiere hablar con administración, marque 2 si quiere hablar con informática...''
\end{notice}

Los campos que se presentan son prácticamente idénticos al IVR genérico:

\noindent\sphinxincludegraphics{{ivr_custom}.png}

La única diferencia es que no existe el campo \textbf{Expresión regular lista negra}, que carece de sentido en este tipo de IVRs.

Pulsando el botón de la siguiente imagen se pueden definir las equivalencias deseadas:

\noindent\sphinxincludegraphics{{ivr_custom2}.png}

En este caso se puede marcar 1, 2 y 3 (todo lo demás será considerado inválido y activará la \textbf{Configuración de Error}):

\noindent\sphinxincludegraphics{{ivr_custom3}.png}
\begin{itemize}
\item {} 
1: Llamada a la extensión interna 200, creada {\hyperref[pbx_features/huntgroups:huntgroups]{\sphinxcrossref{\DUrole{std,std-ref}{en la sección anterior}}}} y que apunta al grupo de salto \emph{Recepción}.

\item {} 
2: Llama a la extensión interna 101.

\item {} 
3: Desvía la llamada al número externo 676 676 676.

\end{itemize}

\begin{notice}{note}{Nota:}
Cada una de las opciones del IVR a medida permite la selección de una locución que pisa a la \textbf{locución de éxito} si está definida. De esta forma aparte de tener una locución de éxito genérica con ``Contactando'' podríamos tener otra que dijera ``Contactando con Administración, espere por favor''.
\end{notice}


\section{Salas de audioconferencias}
\label{pbx_features/conference_rooms:salas-de-audioconferencias}\label{pbx_features/conference_rooms::doc}\label{pbx_features/conference_rooms:conference-rooms}
IvozProvider provee la funcionalidad de salas de audiconferencias que se pueden configurar en la sección \textbf{Configuración de empresa} \textgreater{} \textbf{Salas de conferencias}.
\paragraph{Crear una sala de audioconferencias}

La siguiente imagen ilustra el proceso de creación de una sala de audioconferencias:

\noindent\sphinxincludegraphics{{conference_room}.png}
\begin{description}
\item[{Nombre\index{Nombre|textbf}}] \leavevmode\phantomsection\label{pbx_features/conference_rooms:term-nombre}
Nombre por el que se referenciará la sala en otras secciones

\item[{Límite de participantes\index{Límite de participantes|textbf}}] \leavevmode\phantomsection\label{pbx_features/conference_rooms:term-limite-de-participantes}
Apartir del número especificado, no se admitirán más miembros.

\item[{Protegido con contraseña / Código PIN\index{Protegido con contraseña / Código PIN|textbf}}] \leavevmode\phantomsection\label{pbx_features/conference_rooms:term-protegido-con-contrasena-codigo-pin}
Se puede forzar al sistema a pedir un PIN para poder entrar. En caso de activar, se puede introducir un PIN numérico.

\end{description}

\begin{notice}{note}{Nota:}
Para desactivar el límite de participantes, configurar su valor a 0.
\end{notice}
\paragraph{Asociar una extensión o un DDI externo}

De poco sirve una sala de conferencias si no podemos meter a los participantes a la misma.

Para poder meter participantes, el primer paso sería asociar una extensión interna a la sala, para poder desvíar a la misma, transferir, etc.:

\noindent\sphinxincludegraphics{{conference_room_ext}.png}

En la siguiente sección veremos como también se puede apuntar un {\hyperref[pbx_features/external_ddis:external\string-ddis]{\sphinxcrossref{\DUrole{std,std-ref}{DDI externo}}}} a una sala de conferencias, para meter en la sala a gente externa a la empresa.

\begin{notice}{hint}{Consejo:}
Existen otras formas de meter un participante externo a una sala de audioconferencias sin dedicar un DDI externo: que alguien de dentro transfiera la llamada a la extensión de la sala, acceder por medio de un IVR, etc.
\end{notice}


\section{DDIs externos}
\label{pbx_features/external_ddis:ddis-externos}\label{pbx_features/external_ddis:external-ddis}\label{pbx_features/external_ddis::doc}
En la sección preliminar {\hyperref[external_incoming_calls/configure_ddi:settingup\string-ddi]{\sphinxcrossref{\DUrole{std,std-ref}{Dar de alta un DDI externo}}}} se describió con todo detalle la configuración necesaria para configurar un DDI externo directo a un usuario.

Después de explicar las secciones anteriores, podemos llevar esta configuración un poco más allá:


\subsection{Filtro de entrada}
\label{pbx_features/external_ddis:filtro-de-entrada}
Podemos seleccionar el \textbf{Filtro de entrada externo} configurado en {\hyperref[pbx_features/external_filters:external\string-filters]{\sphinxcrossref{\DUrole{std,std-ref}{la sección anterior}}}}.

\noindent\sphinxincludegraphics{{ddi_edit}.png}


\subsection{Tratamientos}
\label{pbx_features/external_ddis:routing-logics}\label{pbx_features/external_ddis:tratamientos}
Vemos que tenemos más opciones aparte de enviar la llamada, una vez superado los filtros de horario y calendario (y previa locución de bienvenida), directamente a un usuario:
\begin{itemize}
\item {} 
{\hyperref[pbx_features/huntgroups:huntgroups]{\sphinxcrossref{\DUrole{std,std-ref}{Grupos de salto}}}}

\end{itemize}

\noindent\sphinxincludegraphics{{ddi_edit2}.png}
\begin{itemize}
\item {} 
{\hyperref[pbx_features/ivrs:generic\string-ivrs]{\sphinxcrossref{\DUrole{std,std-ref}{IVR Genérico}}}}

\end{itemize}

\noindent\sphinxincludegraphics{{ddi_edit3}.png}
\begin{itemize}
\item {} 
{\hyperref[pbx_features/ivrs:custom\string-ivrs]{\sphinxcrossref{\DUrole{std,std-ref}{IVR a medida}}}}

\end{itemize}

\noindent\sphinxincludegraphics{{ddi_edit4}.png}
\begin{itemize}
\item {} 
{\hyperref[pbx_features/conference_rooms:conference\string-rooms]{\sphinxcrossref{\DUrole{std,std-ref}{Salas de audioconferencias}}}}

\end{itemize}

\noindent\sphinxincludegraphics{{ddi_edit5}.png}

\begin{notice}{hint}{Consejo:}
También podemos apuntar el DDI a un {\hyperref[faxing/index:faxing\string-system]{\sphinxcrossref{\DUrole{std,std-ref}{fax virtual}}}}, pero esto se verá en el siguiente bloque.
\end{notice}


\section{Capturas de llamada}
\label{pbx_features/call_captures:capture-groups}\label{pbx_features/call_captures::doc}\label{pbx_features/call_captures:capturas-de-llamada}
Se entiende por captura de llamada \textbf{la acción de un usuario que escucha} (o se entera de otro modo, vía panel de supervisión, etc.) \textbf{que una extensión está sonando y, desde su terminal, roba dicha llamada}.

IvozProvider soporta dos tipos de captura de llamadas:
\begin{description}
\item[{Captura de llamadas directa\index{Captura de llamadas directa|textbf}}] \leavevmode\phantomsection\label{pbx_features/call_captures:term-captura-de-llamadas-directa}
En las capturas de llamadas directas el usuario indica un código que incluye la extensión del teléfono cuya llamada quiere \emph{robar}. Si el código fuera *95, por ejemplo, marcaría *95101 para capturar una llamada que esté sonando en la extensión 101.

\item[{Captura de llamadas indirecta\index{Captura de llamadas indirecta|textbf}}] \leavevmode\phantomsection\label{pbx_features/call_captures:term-captura-de-llamadas-indirecta}
En las capturas de llamadas indirectas, en cambio, el usuario indica un código y el sistema busca qué teléfono está sonando dentro de sus grupos de captura.

\end{description}


\subsection{Grupos de captura de llamada}
\label{pbx_features/call_captures:grupos-de-captura-de-llamada}
Para poder realizar \textbf{capturas de llamada indirectas}, el usuario que captura tiene que pertener al mismo grupo de captura que el usuario al que pretende capturar.

La sección \textbf{Grupos de captura de llamada} nos permite crear estos grupos y decir qué usuarios pertenecen a ellos:

\noindent\sphinxincludegraphics{{capture_groups}.png}

Como vimos en la sección de {\hyperref[pbx_features/users:users]{\sphinxcrossref{\DUrole{std,std-ref}{Usuarios}}}}, también se puede editar un usuario para editar los grupos de captura a los que pertenece.

\begin{notice}{note}{Nota:}
Un usuario puede pertenecer a más de un grupo de captura, el sistema tendrá en cuenta todos sus grupos.
\end{notice}

Desde el momento en el que un usuario está en un grupo de captura, puede empezar a intentar capturar de forma indirecta pero para ello necesita saber el código de captura.


\subsection{Código de captura de llamada}
\label{pbx_features/call_captures:codigo-de-captura-de-llamada}
IvozProvider permite definir los códigos de captura a 2 niveles:
\begin{itemize}
\item {} 
A nivel de marca en \textbf{Configuración de Marca} \textgreater{} \textbf{Servicios}.

\item {} 
A nivel de empresa en \textbf{Configuración de Empresa} \textgreater{} \textbf{Servicios}.

\end{itemize}

Esto permite que el administrador de marca pueda definir unos códigos genéricos para todas sus empresas y que, las empresas que deseen utilizar otros códigos, puedan redefinirlos.

{\hyperref[pbx_features/services:services]{\sphinxcrossref{\DUrole{std,std-ref}{La siguiente sección}}}} explica todo lo relativo a servicios, que engloba los códigos de captura y otros servicios adicionales a los que se accede marcando códigos que comienzan por *.


\section{Servicios adicionales}
\label{pbx_features/services:services}\label{pbx_features/services::doc}\label{pbx_features/services:servicios-adicionales}
Existen \textbf{servicios especiales} a los que se accede \textbf{marcando códigos} especiales \textbf{desde un terminal} de usuario \textbf{cuando éste está en reposo}.

\begin{notice}{danger}{Peligro:}
Los servicios que se definen en esta sección \textbf{no son accesibles en medio de una conversación}. Se activan \textbf{llamando} a los códigos que se mencionarán, no marcándolos en medio de una conversación.
\end{notice}


\subsection{Listado de códigos a nivel global}
\label{pbx_features/services:listado-de-codigos-a-nivel-global}
En el momento de realizar esta documentación, existen los siguientes \textbf{servicios especiales} visibles en la sección \textbf{Gestión general} \textgreater{} \textbf{Servicios}:

\noindent\sphinxincludegraphics{{services_god}.png}
\begin{description}
\item[{Captura Directa\index{Captura Directa|textbf}}] \leavevmode\phantomsection\label{pbx_features/services:term-captura-directa}
Es el servicio que permite capturar metiendo el código que se asigne seguido de la extensión del teléfono a capturar.

\item[{Captura de Grupo\index{Captura de Grupo|textbf}}] \leavevmode\phantomsection\label{pbx_features/services:term-captura-de-grupo}
Es el servicio que permite capturar el teléfono que esté sonando dentro de tu(s) grupo(s) de captura.

\item[{Consultar el buzón de voz\index{Consultar el buzón de voz|textbf}}] \leavevmode\phantomsection\label{pbx_features/services:term-consultar-el-buzon-de-voz}
Este servicio permite acceder a un menú de voz que te presenta los mensajes de voz nuevos, viejos, etc. Es una alternativa a la recepción de mensajes de voz vía correo electrónico.

\end{description}

A medida que la solución vaya evolucionando y surjan servicios nuevos, aparecerán en este listado para que el operador global sepa de su existencia y lo comunique a sus operadores de marca.

\begin{notice}{attention}{Atención:}
Este listado determina los servicios disponibles y los códigos por defecto de las \textbf{nuevas marcas}.
\end{notice}

\begin{notice}{hint}{Consejo:}
Cambiar un código solo afecta a las marcas que se creen tras el cambio.
\end{notice}


\subsection{Definición de servicios y códigos a nivel de marca}
\label{pbx_features/services:definicion-de-servicios-y-codigos-a-nivel-de-marca}
La sección \textbf{Configuración de Marca ** \textgreater{} **Servicios} permite al operador de marca:
\begin{itemize}
\item {} 
Redefinir el código de acceso por defecto a dichos servicios para las empresas que a su vez no lo redefinan.

\item {} 
Borrar servicios que no quieran que puedan utilizar sus empresas.

\end{itemize}

Por defecto este listado aparece con todas los servicios y los códigos configurados a nivel Global:

\noindent\sphinxincludegraphics{{services_brand_list}.png}

\begin{notice}{attention}{Atención:}
Este listado determina los servicios disponibles y los códigos por defecto de las \textbf{nuevas empresas}.
\end{notice}

\begin{notice}{hint}{Consejo:}
Cambiar un código solo afecta a las empresas que se creen tras el cambio. Borrar un servicio hace que no esté disponible para ninguna empresa de la marca.
\end{notice}


\subsection{Definición de códigos a nivel de empresa}
\label{pbx_features/services:definicion-de-codigos-a-nivel-de-empresa}
Cada empresa puede \emph{pisar} los valores por defecto asignados por su \emph{operador de marca} accediendo a \textbf{Configuración de Empresa} \textgreater{} \textbf{Servicios} y cambiando el código asignado.
\paragraph{Empresa que quiere capturar con ** en lugar de con *95:}

\noindent\sphinxincludegraphics{{services_company_edit}.png}

\begin{notice}{hint}{Consejo:}
Los servicios que el \emph{administrador de empresa} borre no podrán ser utilizados por sus usuarios.
\end{notice}


\section{Grabación de llamadas}
\label{pbx_features/call_recording:grabacion-de-llamadas}\label{pbx_features/call_recording::doc}\label{pbx_features/call_recording:call-recordings}
IvozProvider permite grabar las llamadas que se cursan en 2 modalidades distintas:
\begin{itemize}
\item {} 
\textbf{Grabación automática} para llamadas desde/hacia cierto {\hyperref[pbx_features/external_ddis:external\string-ddis]{\sphinxcrossref{\DUrole{std,std-ref}{DDI externo}}}}.

\item {} 
\textbf{Grabación bajo demanda} solicitada por un usuario en medio de una conversación.

\end{itemize}


\subsection{Grabación automática por DDI}
\label{pbx_features/call_recording:grabacion-automatica-por-ddi}
En el caso de grabaciones automáticas por DDI, \textbf{se graba toda la conversación}: desde el principio hasta el final.

Se distinguen 2 casos:
\begin{itemize}
\item {} 
\textbf{Llamadas entrantes a un DDI}: la grabación seguirá mientras la parte externa de la llamada permanezca en la conversación.

\item {} 
\textbf{Llamadas salientes utilizando un DDI} como {\hyperref[pbx_features/external_ddis:external\string-ddis]{\sphinxcrossref{\DUrole{std,std-ref}{DDI saliente}}}}: mientras el interlocutor externo permanezca en la conversación, la grabación sigue.

\end{itemize}

\begin{notice}{attention}{Atención:}
El hecho de que \textbf{mientras el interlocutor permanezca en la llamada la grabación continúe}, hace que no importe cuántes veces se transfiera la llamada de un usuario a otro, la grabación contendrá toda la conversación (desde el punto de vista del participante externo).
\end{notice}
\paragraph{Grabar todas las llamadas de un DDI}

Basta con editar el DDI en cuestión y habilitar las grabaciones:

\noindent\sphinxincludegraphics{{recordings_ddi}.png}

Existen 4 opciones:
\begin{itemize}
\item {} 
Desactivar la grabación

\item {} 
Activarla para llamadas entrantes a dicho DDI

\item {} 
Activarla para llamadas salientes que presente dicho DDI

\item {} 
Activarla para ambas

\end{itemize}


\subsection{Grabación bajo demanda}
\label{pbx_features/call_recording:grabacion-bajo-demanda}
La grabación bajo demanda u \emph{on-demand} la tiene que activar el \emph{operador de marca} para las empresas que la necesiten, sin más que editar su empresa y configurar el código deseado:

\noindent\sphinxincludegraphics{{recordings_ondemand}.png}

\begin{notice}{warning}{Advertencia:}
A diferencia de los {\hyperref[pbx_features/services:services]{\sphinxcrossref{\DUrole{std,std-ref}{Servicios}}}} comentado en la sección anterior, la funcionalidad de grabación bajo demanda se activa en mitad de una conversación.
\end{notice}


\subsubsection{Activación por medio de tecla \emph{Record}}
\label{pbx_features/call_recording:activacion-por-medio-de-tecla-record}
Los terminales \emph{Yealink} soportan el envío de mensajes \href{https://tools.ietf.org/html/rfc6086}{SIP INFO} con una cabecera \emph{Record} (ver \href{http://www.yealink.com/Upload/document/UsingCallRecordingFeatureonYealinkPhones/UsingCallRecordingFeatureonYealinkSIPT2XPphonesRev\_610-20561729764.pdf}{referencia}). No es un estándar, pero al ser \emph{Yealink} uno de los modelos soportados, IvozProvider incluye soporte para la activación de grabación bajo demanda de esta forma.

\begin{notice}{important}{Importante:}
El código seleccionado no influye en este caso, pero \textbf{la empresa sí que tiene que tener las grabaciones bajo demanda activadas}.
\end{notice}

La activación de las grabaciones es muy simple en este caso, basta con pulsar la tecla y el sistema inicia la grabación.


\subsubsection{Activación por medio de transferencia ciega frustrada}
\label{pbx_features/call_recording:activacion-por-medio-de-transferencia-ciega-frustrada}
Existe otra forma de acceder a esta funcionalidad para los terminales que no tengan soporte para el método anterior.

\begin{notice}{danger}{Peligro:}
Este método de acceder a la funcionalidad es una forma imaginativa de hacerla accesible para terminales sin soporte nativo de tecla \emph{Record} (que es el método recomendado). En función del terminal en cuestión y de la configuración del mismo, resultará más o menos cómodo utilizar la funcionalidad (tecla rápida de transferencia ciega, no retener al interlocutor, etc.).
\end{notice}

Los pasos a seguir en este método alternativo e imaginativo son los siguientes:
\begin{itemize}
\item {} 
No se activa marcando el código en medio de la conversación.

\item {} 
Se activa iniciando una transferencia ciega al código configurado.

\item {} 
El sistema rechazará la transferencia e iniciará la grabación.

\item {} 
El usuario podrá volver a la conversación que tenía (si es que su terminal no ha vuelto solo) y seguir hablando.

\end{itemize}
\paragraph{¿Por qué esta forma tan \emph{peculiar} de activar la grabación y no por medio de tonos normales?}

El motivo de activar la grabación por medio de una transferencia ciega frustrada se debe a la {\hyperref[architecture/index:architecture]{\sphinxcrossref{\DUrole{std,std-ref}{{[}P{]} Arquitectura general de la plataforma}}}} y, más concretamente, al {\hyperref[architecture/index:audioflow]{\sphinxcrossref{\DUrole{std,std-ref}{flujo de audio RTP}}}}.

Lo habitual suele ser activar servicios por medio de pulsaciones en medio de la llamada (lo que se conoce por \href{https://es.wikipedia.org/wiki/Marcaci\%C3\%B3n\_por\_tonos}{tonos DTMF}). Estos tonos suelen viajar por el mismo camino del audio (siguiendo el \href{https://tools.ietf.org/html/rfc4733}{RFC 4733} o como sonido audible).

IvozProvider utiliza la transmisión de estos tonos siguiendo el \href{https://tools.ietf.org/html/rfc4733}{RFC 4733} y, por tanto, dichos paquetes de audio pasan por los \emph{media-relays}, que se limitan a reenviar el audio sin analizar su contenido. Al no analizar su contenido, no pueden detectar las pulsaciones y activar lógicas.

\begin{notice}{note}{Nota:}
Esta decisión de diseño es la que permite escalar la solución y ser capaz de gestionar cientos de miles de llamadas concurrentes.
\end{notice}

Al realizar una transferencia ciega, en cambio, se produce señalización SIP dentro de un diálogo (en concreto, un \href{https://tools.ietf.org/html/rfc3515}{REFER}) y sí que es posible activar lógicas de este estilo, ya que la señalización sigue \sphinxtitleref{otro camino \textless{}signallingflow\textgreater{}} que incluye a los elementos con lógica (servidores de aplicación y \emph{proxies}).

\begin{notice}{warning}{Advertencia:}
La grabación bajo demanda graba la llamada desde que se activa hasta que el usuario que la activó desaparece de la misma.
\end{notice}

\begin{notice}{important}{Importante:}
No existe la funcionalidad de parar la grabación, independientemente del método elegido para activarla.
\end{notice}


\subsection{Listado de grabaciones}
\label{pbx_features/call_recording:listado-de-grabaciones}
El \emph{administrador de empresa} puede acceder a las grabaciones realizadas por medio de la sección \textbf{Configuración de Empresa} \textgreater{} \textbf{Grabaciones}:

\noindent\sphinxincludegraphics{{recordings_list}.png}

Haciendo clic en una de ellas, podría escucharla desde la \emph{web} o descargársela en formato MP3:

\noindent\sphinxincludegraphics{{recordings_list2}.png}

En el caso de una grabación bajo demanda, se indica qué usuario la inició:

\noindent\sphinxincludegraphics{{recordings_list3}.png}


\chapter{Tarificación y facturación}
\label{billing_and_invoices/index::doc}\label{billing_and_invoices/index:tarificacion-y-facturacion}
En este bloque se tratará un tema de primordial importancia para los administradores de marca:
\begin{itemize}
\item {} 
Crear planes de precio para poner precio a las llamadas de los usuarios.

\item {} 
Crear facturas que recojan el detalle y el consumo de sus empresas.

\end{itemize}

Pare ello, abordaremos las siguientes cuestiones:


\section{Histórico de llamadas}
\label{billing_and_invoices/call_registry:historico-de-llamadas}\label{billing_and_invoices/call_registry::doc}\label{billing_and_invoices/call_registry:call-registry}
Los históricos de llamadas muestran todas las llamadas de la plataforma y se muestran a 3 niveles:
\begin{itemize}
\item {} 
\textbf{Gestión general} \textgreater{} \textbf{Histórico de llamadas}

\item {} 
\textbf{Configuración de Marca} \textgreater{} \textbf{Histórico de llamadas}

\item {} 
\textbf{Configuración de Empresa} \textgreater{} \textbf{Histórico de llamadas}

\end{itemize}

En cada uno de los niveles se muestran las llamadas filtradas convenientemente.
\paragraph{A nivel global (god)}

Muestra todas las llamadas de la plataforma, indicando la marca y la empresa de cada una de ellas:

\noindent\sphinxincludegraphics{{call_registry_god}.png}
\paragraph{A nivel de marca}

Muestra todas las llamadas de la marca emulada, indicando la empresa de cada una de ellas:

\noindent\sphinxincludegraphics{{call_registry_brand}.png}
\paragraph{A nivel de empresa}

Muestra todas las llamadas de la empresa emulada:

\noindent\sphinxincludegraphics{{call_registry_company}.png}

\begin{notice}{note}{Nota:}
El exportador a \href{https://es.wikipedia.org/wiki/CSV}{CSV} permite exportar el listado para su posterior almacenamiento o procesado.
\end{notice}

\begin{notice}{hint}{Consejo:}
Si se accede a una llamada concreta, se muestra información adicional sobre la misma. Esta información adicional depende del nivel (\emph{god}, marca o empresa) y provee información sobre desvíos, transferencias, etc.
\end{notice}


\section{Llamadas facturables}
\label{billing_and_invoices/billable_calls:llamadas-facturables}\label{billing_and_invoices/billable_calls::doc}\label{billing_and_invoices/billable_calls:billable-calls}
Los listados de llamadas de las secciones \textbf{Llamadas facturables} muestran solo las \textbf{llamadas que implican coste}.

\begin{notice}{important}{Importante:}
La gran diferencia respecto a las llamadas del {\hyperref[billing_and_invoices/call_registry:call\string-registry]{\sphinxcrossref{\DUrole{std,std-ref}{Histórico de llamadas}}}} es que todas las que aparecen aquí implican coste (no aparecen, por tanto, llamadas internas, etc.)
\end{notice}

Se muestra el coste asociado a las llamadas (una vez calculado) y, dado que las empresas son notificadas de sus llamadas por medio de facturas emitidas por el \emph{operador de marca}, solo se existe a dos niveles:
\begin{itemize}
\item {} 
A nivel global (\emph{god}).

\item {} 
A nivel de marca.

\end{itemize}

Estos listados muestran la siguiente información:
\begin{description}
\item[{Fecha\index{Fecha|textbf}}] \leavevmode\phantomsection\label{billing_and_invoices/billable_calls:term-fecha}
Fecha y hora del establecimiento de la llamada.

\item[{Marca\index{Marca|textbf}}] \leavevmode\phantomsection\label{billing_and_invoices/billable_calls:term-marca}
Solo visible a nivel \emph{god}, indica la marca de la empresa en cuestión.

\item[{Compañía\index{Compañía|textbf}}] \leavevmode\phantomsection\label{billing_and_invoices/billable_calls:term-compania}
Empresa responsable de la llamada.

\item[{Destino\index{Destino|textbf}}] \leavevmode\phantomsection\label{billing_and_invoices/billable_calls:term-destino}
Número externo al que se ha llamado.

\item[{Patrón de destino\index{Patrón de destino|textbf}}] \leavevmode\phantomsection\label{billing_and_invoices/billable_calls:term-patron-de-destino}
Indica el {\hyperref[external_outgoing_calls/noplan_nocall:price\string-pattern]{\sphinxcrossref{\DUrole{std,std-ref}{patrón de precio}}}} en base al cual se ha puesto precio a la llamada.

\item[{Duración\index{Duración|textbf}}] \leavevmode\phantomsection\label{billing_and_invoices/billable_calls:term-duracion}
Indica cuánto ha durado la llamada.

\item[{Tarificado (sí/no)\index{Tarificado (sí/no)|textbf}}] \leavevmode\phantomsection\label{billing_and_invoices/billable_calls:term-tarificado-si-no}
Indica si el proceso que pone precio a las llamadas ha calculado el precio de esta llamada concreta.

\item[{Precio\index{Precio|textbf}}] \leavevmode\phantomsection\label{billing_and_invoices/billable_calls:term-precio}
Coste calculado para la llamada.

\item[{Plan de precio\index{Plan de precio|textbf}}] \leavevmode\phantomsection\label{billing_and_invoices/billable_calls:term-plan-de-precio}
{\hyperref[external_outgoing_calls/noplan_nocall:price\string-plan]{\sphinxcrossref{\DUrole{std,std-ref}{Plan de precio}}}} en base al cual se ha puesto precio a la llamada.

\item[{Contrato de peering\index{Contrato de peering|textbf}}] \leavevmode\phantomsection\label{billing_and_invoices/billable_calls:term-contrato-de-peering}
Indica por qué {\hyperref[external_incoming_calls/peering_contracts:peering\string-contracts]{\sphinxcrossref{\DUrole{std,std-ref}{Contrato de peering}}}} ha salido la llamada.

\item[{Factura\index{Factura|textbf}}] \leavevmode\phantomsection\label{billing_and_invoices/billable_calls:term-factura}
Indica si la llamada está incluida en alguna {\hyperref[billing_and_invoices/invoices:invoices]{\sphinxcrossref{\DUrole{std,std-ref}{factura}}}}.

\item[{Tipo (entrante/saliente)\index{Tipo (entrante/saliente)|textbf}}] \leavevmode\phantomsection\label{billing_and_invoices/billable_calls:term-tipo-entrante-saliente}
Dado que ciertas llamadas entrantes pueden implicar coste (ver {\hyperref[external_incoming_calls/configure_ddi:bill\string-inbound]{\sphinxcrossref{\DUrole{std,std-ref}{tarificación de llamadas entrantes}}}}), indica si la llamada es entrante o saliente.

\end{description}

\begin{notice}{note}{Nota:}
Las llamadas aparecen en este listado en cuanto se cuelgan. Pasados unos minutos, el proceso que pone precios a las llamadas habrá tarificado la llamada (\emph{Tarificado} igual a `Sí') y tendremos disponible el \textbf{Precio} calculado.
\end{notice}


\section{Planes de precio}
\label{billing_and_invoices/pricing_plans:planes-de-precio}\label{billing_and_invoices/pricing_plans::doc}
En la sección {\hyperref[external_outgoing_calls/noplan_nocall:noplan\string-nocall]{\sphinxcrossref{\DUrole{std,std-ref}{Sin plan de precio, no hay llamada}}}} se hacía una introducción bastante completa sobre el proceso manual de creación de un plan de precios y los conceptos más importantes:
\begin{itemize}
\item {} 
\textbf{Un plan de precios agrupa un listado de patrones de precio (prefijos de llamada) con sus detalles de precio}:
\begin{itemize}
\item {} 
Precio por minuto

\item {} 
Establecimiento de llamada

\item {} 
Facturación por segundos / minutos /etc.

\end{itemize}

\item {} 
Un plan de precios se asocia a una empresa concreta, indicando el período de validez de dicho plan.

\item {} 
Una empresa podía tener varios planes de precios en un momento concreto para una llamada concreta.

\item {} 
En este último caso, el coste de la llamada se calcularía utilizando el plan de precio de menor métrica.

\end{itemize}


\subsection{Creación manual}
\label{billing_and_invoices/pricing_plans:creacion-manual}
La {\hyperref[external_outgoing_calls/noplan_nocall:price\string-plan]{\sphinxcrossref{\DUrole{std,std-ref}{creación manual de un plan de precio}}}} implicaba la creación previa de un {\hyperref[external_outgoing_calls/noplan_nocall:price\string-pattern]{\sphinxcrossref{\DUrole{std,std-ref}{patrón de precio}}}}.

En ese momento, es posible que el futuro administrador de marca se haya dado cuenta de la titánica tarea que implicaría crear miles de patrones de precio (254 países por las distintas redes móviles, fijos, numeraciones especiales, etc.) para luego poder agruparlos en un plan de precios.

Es por ello que el proceso de creación de planes y patrones de precio se realiza partiendo de un \href{https://es.wikipedia.org/wiki/CSV}{CSV}.


\subsection{Importación vía CSV}
\label{billing_and_invoices/pricing_plans:importacion-via-csv}
El primer paso es crear un plan de precios vacío sobre el que importar nuestros precios (sección \textbf{Configuración de marca} \textgreater{} \textbf{Planes de precio}):

\noindent\sphinxincludegraphics{{pricing_plans_add}.png}

Accedemos al listado (vacío) del plan de precio que acabamos de crear:

\noindent\sphinxincludegraphics{{pricing_plans_add_price}.png}

El botón clave para este proceso de importación masiva es el siguiente:

\noindent\sphinxincludegraphics{{pricing_plan_csv}.png}

Una vez elegido el archivo a importar, se nos presenta la siguiente ventana:

\noindent\sphinxincludegraphics{{pricing_plan_csv3}.png}

En esta ventana podríamos seleccionar qué contiene cada columna, en caso de no haber creado el \href{https://es.wikipedia.org/wiki/CSV}{CSV} en el formato recomendado. Del mismo modo, se nos ofrece la posibilidad de ignorar la primera línea, en caso de que incluya los nombres de las columnas en lugar de datos.

\begin{notice}{hint}{Consejo:}
El proceso de importación se realiza en segundo plano, permitiendo al administrador de marca seguir configurando otros aspectos de la plataforma mientras se completa.
\end{notice}


\subsubsection{Formato CSV}
\label{billing_and_invoices/pricing_plans:formato-csv}
A pesar de que la ventana anterior nos permite importar archivos \href{https://es.wikipedia.org/wiki/CSV}{CSV} en distintos formatos, lo mejor es importar un archivo en el formato adecuado para simplicar este proceso.

El formato del archivo \href{https://es.wikipedia.org/wiki/CSV}{CSV} está explicado en la propia sección de ayuda contextual, que incluye un enlace para poder descargar un archivo de ejemplo:

\noindent\sphinxincludegraphics{{pricing_plan_csv2}.png}

El formato, por lo tanto, tiene que ser:
\begin{itemize}
\item {} 
Nombre del patrón de precio

\item {} 
Descripción del patrón de precio

\item {} 
Prefijo

\item {} 
Precio por minuto

\item {} 
Precio de establecimiento

\item {} 
Período de facturación

\end{itemize}

\begin{notice}{note}{Nota:}
Los números decimales tienen que estar entrecomillados con comillas dobles y tienen que usar la coma como separador decimal.
\end{notice}

\begin{notice}{important}{Importante:}
El sistema de importación creará los patrones de precio que sean necesarios. Si ya existe un patrón de precios con ese prefijo, no se creará, simplemente se vinculará.
\end{notice}

\begin{notice}{warning}{Advertencia:}
\textbf{El período de facturación determina cada cuántos segundos se incrementa el precio de la llamada}.
\begin{itemize}
\item {} 
Si se pone a 1, implica una tarificación por segundos y cada segundo implicará un coste que será el \emph{precio por minuto} dividido entre 60.

\item {} 
Para facturación por minutos, habrá que poner 60 en este campo y, cada bloque de 1 minuto se sumará el \emph{precio por minuto} al precio final.

\end{itemize}
\end{notice}

Una vez completada la importación, solo faltaría asociar el nuevo plan de precios a las empresas que queramos, siguiendo {\hyperref[external_outgoing_calls/noplan_nocall:pricing\string-plan\string-to\string-company]{\sphinxcrossref{\DUrole{std,std-ref}{el procedimiento explicado en el bloque anterior}}}}.


\section{Tarificación de llamadas}
\label{billing_and_invoices/bill_a_call::doc}\label{billing_and_invoices/bill_a_call:tarificacion-de-llamadas}
Tarificar una llamada es la \textbf{acción de poner precio} a una llamada que implica coste.


\subsection{Tarificación automática}
\label{billing_and_invoices/bill_a_call:tarificacion-automatica}
Tal y como se ha explicado con anterioridad, el proceso de tarificación es automático:
\begin{itemize}
\item {} 
En el momento en el que una llamada se va a establecer, se verifica que se vaya a poder tarificar.
\begin{itemize}
\item {} 
Si no se va a poder tarificar atendiendo a los planes de precios activos para la empresa, la llamada no se cursará.

\end{itemize}

\item {} 
En el momento en el que una llamada que implique coste se cuelga, aparece en el listado de {\hyperref[billing_and_invoices/billable_calls:billable\string-calls]{\sphinxcrossref{\DUrole{std,std-ref}{Llamadas facturables}}}}.

\item {} 
Pasados unos minutos, el proceso tarificador evaluará las llamadas tarificables sin coste y rellenará los siguientes campos del registro anterior:
\begin{itemize}
\item {} 
Precio

\item {} 
Plan de precio

\item {} 
Patrón de destino (que es en realidad el patrón de precio)

\item {} 
Tarificado a `Sí'

\end{itemize}

\end{itemize}


\subsection{Retarificación manual}
\label{billing_and_invoices/bill_a_call:retarificacion-manual}
Puede ocurrir que una llamada se tarifique de forma incorrecta, por múltiples motivos:
\begin{itemize}
\item {} 
Plan de precio importado con una errata.

\item {} 
Múltiples planes de precios con métricas incorrectas.

\item {} 
Plan de precio adicional sin vincular.

\item {} 
Etc.

\end{itemize}

Para estos casos, el \emph{administrador de marca} puede re-tarificar las llamadas que considere mal tarificadas.

\begin{notice}{important}{Importante:}
Retarificar una llamada consiste en volver a calcular el precio de la misma. Lógicamente, la retarificación se calculará en el momento solicitado, teniendo en cuenta las configuración actual de planes de precio (y no las del momento del establecimiento de la llamada).
\end{notice}

Para retarificar, basta con seleccionar las llamadas en \textbf{Configuración de Marca} \textgreater{} \textbf{Llamada facturables} y presionar el botón \textbf{Tarificar llamadas}:

\noindent\sphinxincludegraphics{{retarificate}.png}

\begin{notice}{error}{Error:}
\textbf{No se puede retarificar una llamada que esté incluida en una factura}. Es decir, si la llamada seleccionada tiene el campo \textbf{Factura} con valor, habrá que borrar esta factura previamente. El motivo es estar seguros que no existen facturas con llamadas mal tarificadas: si retarificas una llamada, regeneras la factura que la contiene.
\end{notice}


\section{Generación de facturas}
\label{billing_and_invoices/invoices:invoices}\label{billing_and_invoices/invoices::doc}\label{billing_and_invoices/invoices:generacion-de-facturas}
El objetivo final de todo el proceso de facturación es generar facturas que incluyan las llamadas con coste de una empresa concreta.


\subsection{Plantillas de facturas}
\label{billing_and_invoices/invoices:plantillas-de-facturas}
Antes de generar una factura de ejemplo, es importante entender que las facturas generadas utilizan unas plantillas que permiten su modificación.

\begin{notice}{hint}{Consejo:}
De este modo, cada \emph{operador de marca} puede crear plantillas con los datos deseados, la estética deseada, añadir logos y hasta gráficas de consumo.
\end{notice}

Todo ello se realiza utilizando la librería \href{https://github.com/psliwa/PHPPdf}{PHPPdf}.

La ayuda contextual de la sección \textbf{Configuración de Marca} \textgreater{} \textbf{Plantillas de facturas} incluye una explicación resumida del proceso de creación de plantillas. En la \href{https://github.com/psliwa/PHPPdf}{página oficial de PHPPdf} se puede encontrar más información.

Por defecto, IvozProvider incluye las siguientes plantillas de ejemplo (en la ayuda contextual se pueden encontrar enlaces a las mismas):

\noindent\sphinxincludegraphics{{invoices_templates}.png}


\subsection{Costes fijos}
\label{billing_and_invoices/invoices:costes-fijos}
Los costes fijos son un concepto fijo que se pueden añadir a las facturas que utilicen plantillas que tengan en cuenta costes fijos.

Sirva la siguiente imagen de ejemplo (sección \textbf{Costes Fijos}):

\noindent\sphinxincludegraphics{{fixed_costs}.png}

A la hora de generar una factura, como se verá más adelante, se podrá indicar cuáles de estos conceptos se incluyen en la factura (y en qué cantidades).


\subsection{Creacion de una factura}
\label{billing_and_invoices/invoices:creacion-de-una-factura}
La sección \textbf{Facturas} es la que permite al \textbf{operador de marca} generar facturas para emitir a sus empresas.

Añadimos una factura nueva para explicar el proceso:

\noindent{\hspace*{\fill}\sphinxincludegraphics{{invoice_add}.png}\hspace*{\fill}}
\begin{description}
\item[{Número\index{Número|textbf}}] \leavevmode\phantomsection\label{billing_and_invoices/invoices:term-numero}
Será incluído en la factura y representa el número de factura

\item[{Empresa\index{Empresa|textbf}}] \leavevmode\phantomsection\label{billing_and_invoices/invoices:term-empresa}
Empresa para la cual estamos generando la factura

\item[{Fecha inicio/fin\index{Fecha inicio/fin|textbf}}] \leavevmode\phantomsection\label{billing_and_invoices/invoices:term-fecha-inicio-fin}
Tramo temporal cuyas llamadas queremos tener en cuenta

\item[{Impuesto\index{Impuesto|textbf}}] \leavevmode\phantomsection\label{billing_and_invoices/invoices:term-impuesto}
Impuesto a añadir al coste total calculado

\item[{Plantilla\index{Plantilla|textbf}}] \leavevmode\phantomsection\label{billing_and_invoices/invoices:term-plantilla}
Plantilla que queremos utilizar para generar esta factura

\end{description}

Añadamos ahora unos costes fijos a esta factura concreta pulsando:

\noindent{\hspace*{\fill}\sphinxincludegraphics{{invoice_add2}.png}\hspace*{\fill}}

Y añadamos los costes fijos que queramos, así como sus cantidades:

\noindent{\hspace*{\fill}\sphinxincludegraphics{{invoice_add3}.png}\hspace*{\fill}}

En este punto, podemos generar la factura pulsando:

\noindent{\hspace*{\fill}\sphinxincludegraphics{{invoice_add4}.png}\hspace*{\fill}}

Pulsando el siguiente botón podemos ver las llamadas que han sido incluidas en la factura:

\noindent{\hspace*{\fill}\sphinxincludegraphics{{invoice_add5}.png}\hspace*{\fill}}

Y pulsando este botón podemos descargar la factura en formato PDF:

\noindent{\hspace*{\fill}\sphinxincludegraphics{{invoice_add6}.png}\hspace*{\fill}}

\begin{notice}{warning}{Advertencia:}
La fecha fin tiene que ser una fecha ya pasada. Es decir, no se puede sacar facturas de tramos futuros o del día actual.
\end{notice}

\begin{notice}{error}{Error:}
Todas las llamadas del tramo escogido tienen que estar facturadas para poder emitir la factura.
\end{notice}


\chapter{{[}P{]} Provisión de terminales}
\label{provisioning/index:p-provision-de-terminales}\label{provisioning/index::doc}\label{provisioning/index:provisioning}

\section{Esquema de provisión}
\label{provisioning/index:esquema-de-provision}

\section{Terminales soportados}
\label{provisioning/index:terminales-soportados}

\section{Sistema de plantillas}
\label{provisioning/index:sistema-de-plantillas}

\chapter{Faxing Virtual}
\label{faxing/index::doc}\label{faxing/index:faxing-system}\label{faxing/index:faxing-virtual}
La solución de \emph{faxing} virtual incluida en IvozProvider es muy simple, pero permite:
\begin{itemize}
\item {} 
Enviar faxes partiendo de un PDF.

\item {} 
Recibir faxes vía correo electrónico y visualización web.

\end{itemize}

\begin{notice}{error}{Error:}
IvozProvider utiliza \href{http://www.voip-info.org/wiki/view/T.38}{T.38} para la emisión y recepción de faxes. Es responsabilidad de el administrador de marca disponer de Contratos de Peering con operadores que soporten dicho protocolo, así como configurar las rutas de salida para utilizar dicho operador.
\end{notice}


\section{Creación de un fax virtual}
\label{faxing/index:creacion-de-un-fax-virtual}
Esta es la interfaz que nos encontramos al crear un nuevo fax en la sección \textbf{Configuración de empresa} \textgreater{} \textbf{Faxes Virtuales}:

\noindent{\hspace*{\fill}\sphinxincludegraphics{{fax_add}.png}\hspace*{\fill}}

Los campos son prácticamente auto-explicativos:
\begin{description}
\item[{Nombre\index{Nombre|textbf}}] \leavevmode\phantomsection\label{faxing/index:term-nombre}
Nombre por el que se referenciará el fax en otras secciones

\item[{Email\index{Email|textbf}}] \leavevmode\phantomsection\label{faxing/index:term-email}
Email donde se recibirán los mails (en caso de marcar `Enviar por email' a `Sí')

\item[{DDI de salida\index{DDI de salida|textbf}}] \leavevmode\phantomsection\label{faxing/index:term-ddi-de-salida}
Número que se utilizará como origen en los faxes salientes

\end{description}

Para recibir faxes en dicha numeración, será necesario apuntarla a nuestro nuevo fax, editando el DDI en la sección \textbf{DDIs}:

\noindent{\hspace*{\fill}\sphinxincludegraphics{{fax_ddi}.png}\hspace*{\fill}}

Para enviar faxes por una ruta concreta (que tenemos probada y sabemos que es óptima para la emisión de faxes), se puede definir una ruta exclusiva para faxes:

\noindent{\hspace*{\fill}\sphinxincludegraphics{{fax_routes}.png}\hspace*{\fill}}

Todos los faxes que envíe esa empresa (para todos los faxes que vaya creando), se enviará por esta ruta.

\begin{notice}{note}{Nota:}
Si se definieran más rutas de faxing, se utilizarían todas siguiendo las lógicas de \emph{load-balancing} y \emph{failover} descritas en {\hyperref[external_outgoing_calls/call_routing:routes\string-metrics]{\sphinxcrossref{\DUrole{std,std-ref}{secciones anteriores}}}}.
\end{notice}

\begin{notice}{important}{Importante:}
Si una empresa no dispone de rutas de faxing, saldrá siguiendo las lógicas de rutado de las llamadas.
\end{notice}


\section{Emitir un fax}
\label{faxing/index:emitir-un-fax}
\noindent{\hspace*{\fill}\sphinxincludegraphics{{fax_send2}.png}\hspace*{\fill}}

\noindent{\hspace*{\fill}\sphinxincludegraphics{{fax_send}.png}\hspace*{\fill}}


\section{Visualización de faxes entrantes}
\label{faxing/index:visualizacion-de-faxes-entrantes}
Los faxes entrantes se pueden recibir vía correo electrónico, pero también pueden ser visualizados y descargados desde el panel web pulsando:

\noindent\sphinxincludegraphics{{fax_list}.png}


\chapter{{[}P{]} Portal de Usuario}
\label{userportal/index:userportal}\label{userportal/index::doc}\label{userportal/index:p-portal-de-usuario}

\section{Credenciales de acceso}
\label{userportal/index:credenciales-de-acceso}

\section{URLs de acceso}
\label{userportal/index:urls-de-acceso}

\section{Funcionalidades}
\label{userportal/index:funcionalidades}

\subsection{Estado del registro SIP}
\label{userportal/index:estado-del-registro-sip}

\subsection{Configuración básica a nivel de usuario}
\label{userportal/index:configuracion-basica-a-nivel-de-usuario}

\chapter{{[}S{]} Elementos de seguridad}
\label{security/index::doc}\label{security/index:s-elementos-de-seguridad}

\section{Firewall con geoIP}
\label{security/index:firewall-con-geoip}

\section{Rangos de IP autorizados por empresa}
\label{security/index:rangos-de-ip-autorizados-por-empresa}
En el proceso de creación de empresas nos saltamos deliberademente un mecanismo de seguridad que \textbf{limita las direcciones IP o rangos de red que pueden utilizar las credenciales de los terminales de una empresa concreta}.

Se puede activar en la sección \textbf{Configuración de Marca} \textgreater{} \textbf{Empresas}:

\noindent{\hspace*{\fill}\sphinxincludegraphics{{authorized_ips2}.png}\hspace*{\fill}}

Todo usuario que quiera conectarse desde una red no incluida no podrá, a pesar de disponer de unas credenciales válidas.

\begin{notice}{error}{Error:}
Una vez activado el filtrado, \textbf{es obligatorio} añadir redes o direcciones válidas o, por el contrario, todas las llamadas se rechazarán:
\end{notice}

\noindent\sphinxincludegraphics{{authorized_ips}.png}

Se pueden añadir direcciones IP y rangos de direcciones, en formato CIDR (IP/mask):

\noindent\sphinxincludegraphics{{authorized_ips3}.png}

\begin{notice}{important}{Importante:}
Este mecanismo limita los orígenes de los usuarios de una empresa, no filtra en absoluto los orígenes de los \textbf{Contratos de Peering}.
\end{notice}


\section{Anti-flooding}
\label{security/index:anti-flooding}
IvozProvider incorpora un mecanismo de \emph{anti-flooding} que evita que un emisor sature nuestra plataforma enviando peticiones. Ambos \emph{proxies} (usuarios y salida) incorporan este mecanismo, que \textbf{limita el número de peticiones desde un dirección origen en un tramo concreto de tiempo}.

\begin{notice}{warning}{Advertencia:}
Cuando un origen llega al límite, el proxy dejará de contestarle durante un tiempo dado. Pasado ese tiempo, volverá a contestarle con normalidad.
\end{notice}

Ciertos orígenes que están automáticamente excluidos de este mecanismo de \emph{anti-flooding}:
\begin{itemize}
\item {} 
Servidores de aplicación de la plataforma.

\item {} 
IPs o rangos autorizados de empresas (ver sección anterior).

\end{itemize}

El operador global puede añadir otras direcciones que queden excluidas de este mecanismo por medio del apartado \textbf{Configuración global} \textgreater{} \textbf{IPs de confianza}:

\noindent\sphinxincludegraphics{{trusted_ips}.png}


\section{Límite de llamadas concurrentes}
\label{security/index:limite-de-llamadas-concurrentes}
Otro mecanismo de seguridad que puede evitar que unas credenciales comprometidas sean utilizadas para establecer cientos de llamadas en poco tiempo, es el mecanismo que \textbf{limita el número de llamadas externas} de cada empresa.

\begin{notice}{note}{Nota:}
Este mecanismo tiene en cuenta los canales externos concurrentes, es decir, cuenta llamadas externas entrantes y llamadas externas salientes.
\end{notice}

Se puede configurar editando una empresa y fijando el valor del siguiente campo:

\noindent{\hspace*{\fill}\sphinxincludegraphics{{call_limit}.png}\hspace*{\fill}}

\begin{notice}{tip}{Truco:}
Para desactivar este mecanismo, basta con fijar el valor a 0.
\end{notice}


\chapter{{[}P{]} Mantenimiento de la solución}
\label{maintenance/index:p-mantenimiento-de-la-solucion}\label{maintenance/index::doc}
A continuación se describen las herramientas que incorpora IvozProvider para facilitar el mantenimiento de la solución:


\section{Capturador de tráfico SIP: Homer}
\label{maintenance/sip_captures:capturador-de-trafico-sip-homer}\label{maintenance/sip_captures::doc}

\section{Visor de logs: Graylog}
\label{maintenance/log_viewer:visor-de-logs-graylog}\label{maintenance/log_viewer::doc}

\section{Gráficas de la plataforma: Grafana}
\label{maintenance/platform_graphs::doc}\label{maintenance/platform_graphs:graficas-de-la-plataforma-grafana}


\renewcommand{\indexname}{Índice}
\printindex
\end{document}
